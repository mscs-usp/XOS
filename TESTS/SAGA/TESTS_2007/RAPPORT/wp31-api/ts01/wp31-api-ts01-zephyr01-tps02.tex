%%%%%%%%%%%%%%%%%%%%%%%%%%%%%%%%%%%%%%%%%
% Template for describing the test procedure specification.
% structure and annotations are quoted from IEEE Std 829-1998
%%%%%%%%%%%%%%%%%%%%%%%%%%%%%%%%%%%%%%%%%



%\subsection{Test Procedure Specification: File Management}
\starttps{wp31-api-ts01-zephyr01-tps02}
%Specify the unique identifier assigned to this test procedure specification.
%\noindent\textbf{Test log identifier: wp31-api-ts01-zephyr01-tl02}

%\noindent\textbf{Test design specification reference: wp31-api-ts01-zephyr01-tds02} 

%Purpose:
%To specify the steps for executing a set of test cases or, more generally, the steps used to analyze a software
%item in order to evaluate a set of features.

%Outline:
%A test procedure specification shall have the following structure:
%a) Test procedure specification identifier.
%b) Purpose;
%c) Special requirements;
%d) Procedure steps.
%The sections shall be ordered in the specified sequence. Additional sections, if required, may be included at
%the end. If some or all of the content of a section is in another document, then a reference to that material
%may be listed in place of the corresponding content. The referenced material must be attached to the test
%procedure specification or available to users of the procedure specification.
%Details on the content of each section are contained in the following subclauses.


\subsubsection{Purpose}
%Describe the purpose of this procedure. If this procedure executes any test cases, provide a reference for each
%of them.
%In addition, provide references to relevant sections of the test item documentation (e.g., references to usage
%procedures).

This procedure executes the test case \refexpdoc{wp31-api-ts01-zephyr01-tcs02}.

%\subsubsection{Special Requirements}
%Identify any special requirements that are necessary for the execution of this procedure. These may include
%prerequisite procedures, special skills requirements, and special environmental requirements.


\subsubsection{Procedure Steps}
%Include the following steps as applicable.

%\paragraph{Log}
%Describe any special methods or formats for logging the results of test execution, the incidents observed, and
%any other events pertinent to the test (see Clauses 9 and 10).

\paragraph{Set Up}
%Describe the sequence of actions necessary to prepare for execution of the procedure.

%The SAGA library is supposed already installed. If not,
%please refer to the previous test to test/validate this
%prerequisite step.


 
The generic test procedure setup includes:
\begin{itemize}
\item Installation and compilation of SAGA.
\item Compiling and running the test program
\end{itemize}

Installation and compilation of SAGA on Debian Linux etch
distribution including the following steps:
\begin{enumerate}
 \item Download of Boost 1.34.1 
 \item Compilation of Boost and manual linking of  a shared library.
 \item Checking out the 0.6 release of SAGA.
 \item Compilation of SAGA.
\end{enumerate}



\paragraph{Start}
%Describe the actions necessary to begin execution of the procedure.


In order to start the test one needs to perform the following steps:
\begin{enumerate}
\item Installation and compilation of SAGA on Debian Linux etch
distribution : these steps are described in the previous test plan
      \refexpdoc{wp31-api-ts01-zephyr01-tps01}.
\item Compile the following simple test program that performs file operation:

\begin{lstlisting}
#include <iostream>
#include <saga.hpp>

saga::file  write_file(saga::directory dir,char *name, int size) {
  char*
 something="**********...************;

  saga::file file;

  file=dir.open(name,saga::file::Write);
  for (int s;size>0;size-=1024) {
    s=1024;
    if (size<1024) {
      s=size;
    }
     file.write(something,1);
  }
  file.close();

  return file;
}



/////////////////////////////////
int main (int argc, char* argv[])
{
  
  printf ("\n\n\n cleaning Working directory....\n");
  system("rm -rf /tmp/SAGA");
  
   // creation of SAGA dir
  printf ("\n\n\n creating directory tree....\n");
  saga::directory dir("any://localhost/tmp/SAGA/",saga::file::ReadWrite|saga::file::Create|saga::file::CreateParents);
  saga::directory dir1k("any://localhost/tmp/SAGA/sup1k",
			saga::file::ReadWrite|saga::file::Create|saga::file::CreateParents);
  saga::directory dir1M("any://localhost/tmp/SAGA/sup1k/sup1M",
			saga::file::ReadWrite|saga::file::Create|saga::file::CreateParents);
  saga::directory dir5M("any://localhost/tmp/SAGA/sup1k/sup1M/sup5M",
			saga::file::ReadWrite|saga::file::Create|saga::file::CreateParents);

  system("ls -Rl /tmp/SAGA");

  // create file of different size
  printf ("\n\n\n creating files....\n");
  saga::file file=write_file(dir,"es.txt",5);
  write_file(dir1k,"es1k.txt",1024);
  write_file(dir1M,"es1M.txt",1024000);
  write_file(dir5M,"es5M.txt",1024000*5);
  write_file(dir5M,"es100M.txt",1024000*100);
  system("ls -Rl /tmp/SAGA");

  

  //copy it
  printf ("\n\n\n copying file SAGA/es.txt to SAGA/as.txt....\n");
  file.copy("/tmp/SAGA/as.txt");
  system("ls -Rl /tmp/SAGA");

  // rename it
  printf ("\n\n\n renaming file SAGA/es.txt in SAGA/es2.txt....\n");
  file.move("/tmp/SAGA/es2.txt");
  system("ls -Rl /tmp/SAGA");

  // erase copy
  printf ("\n\n\n deleting file SAGA/as.txt ....\n");
  saga::file file2("/tmp/SAGA/as.txt");
  file2.remove();
  system("ls -Rl /tmp/SAGA");

  return 0;
}
\end{lstlisting}

using the following makefile
\begin{lstlisting}
#  Copyright (c) 2005-2006 Andre Merzky (andre@merzky.net)
SAGA_ROOT=/home/kortas/SOFTS/BUILD/saga-c++-0.6-src

SAGA_SRC         = main.cpp
SAGA_BIN         = main

SAGA_ADD_BIN_OBJ = $(SAGA_SRC:%.cpp=%.o)

include $(SAGA_ROOT)/make/saga.application.mk
\end{lstlisting}

 \item Finally run the test program
\begin{lstlisting}
 export LD_LIBRARY_PATH=/home/kortas/SOFTS/saga-06/lib:\
	/home/kortas/SOFTS/boost_34_1/lib:$LD_LIBRARY_PATH:
 export SAGA_LOCATION=/home/kortas/SOFTS/saga-06/
 ./main
\end{lstlisting}
\end{enumerate}

As a comparison element, here is the equivalent shell script. 
\label{api:file_script}
\begin{lstlisting}
echo cleaning Working directory....
\rm -rf /tmp/SAGA

echo
echo
echo creating directory tree....
mkdir -p /tmp/SAGA/sup1k/sup1M/sup5M
ls -l -R /tmp/SAGA 

# creation source file
python -c "print '*'*1024*100" > /tmp/source

# creation 10, 1k, 1M and 5M files

echo
echo
echo creating files...

dd count=1 bs=1024 if=/tmp/source  of=/tmp/SAGA/es.txt
dd count=1 bs=1k   if=/tmp/source  of=/tmp/SAGA/sup1k/es1k.txt
dd count=1 bs=1M   if=/tmp/source  of=/tmp/SAGA/sup1k/sup1M/es1M.txt
dd count=1 bs=5M   if=/tmp/source  of=/tmp/SAGA/sup1k/sup1M/sup5M/es5M.txt

ls -l -R /tmp/SAGA 

echo
echo
echo copying�file SAGA/es.txt to SAGA/as.txt....
cp /tmp/SAGA/es.txt  /tmp/SAGA/as.txt 
ls -l -R /tmp/SAGA

echo
echo
echo renaming�file SAGA/es.txt in SAGA/es2.txt....
mv /tmp/SAGA/es.txt  /tmp/SAGA/es2.txt 
ls -l -R /tmp/SAGA

echo
echo
echo deleting�file SAGA/as.txt
rm /tmp/as.txt

ls -l -R /tmp/SAGA
\end{lstlisting}

%\paragraph{Proceed}
%Describe any actions necessary during execution of the procedure.

%\paragraph{Measure}
%Describe how the test measurements will be made (e.g., describe how remote terminal response time is to be
%measured using a network simulator).

%\paragraph{Shut Down}
%Describe the actions necessary to suspend testing, when unscheduled events dictate.

%\paragraph{Restart}
%Identify any procedural restart points and describe the actions necessary to restart the procedure at each of
%these points.

%\paragraph{Stop}
%Describe the actions necessary to bring execution to an orderly halt.


%\paragraph{Wrap Up}
%Describe the actions necessary to restore the environment.


%\paragraph{Contingencies}
%Describe the actions necessary to deal with anomalous events that may occur during execution.
%\label{api:contingencies}


