%%%%%%%%%%%%%%%%%%%%%%%%%%%%%%%%%%%%%%%%%
% Template for describing the test plan.
% structure and annotations are quoted from IEEE Std 829-1998
%%%%%%%%%%%%%%%%%%%%%%%%%%%%%%%%%%%%%%%%%

%\subsection{Test Plan}
\starttp{wp31-api-ts01-tp}{Evaluation of the SAGA API}
%Specify the unique identifier assigned to this test plan.
%\noindent\textbf{Test plan identifier: wp31-api-ts01-tp}%insert file name here without extension '.tex'

%Purpose:
%To prescribe the scope, approach, resources, and schedule of the testing activities. To identify the items
%being tested, the features to be tested, the testing tasks to be performed, the personnel responsible for each
%task, and the risks associated with this plan.

\subsubsection{Introduction}
%Summarize the software items and software features to be tested. The need for each item and its history may
%be included.
%References to the following documents, when they exist, are required in the highest level test plan:
%a) Project authorization;
%b) Project plan;
%c) Quality assurance plan;
%d) Configuration management plan;
%e) Relevant policies;
%f) Relevant standards.
%In multilevel test plans, each lower-level plan must reference the next higher-level plan.

This test plan evaluates the completeness of the C++ implementation of
SAGA API developed
by WP3.1. The SAGA functionalities include:

\begin{itemize}

	\item Submission, monitoring, deletion of jobs.

	\item Creation, deletion, copying and renaming of files.

	\item Remote procedure Calls.

	\item Stream management.

	\item  Replica management
\end{itemize}


\subsubsection{Test Items}
%Identify the test items including their version/revision level. Also specify characteristics of their transmittal
%media that impact hardware requirements or indicate the need for logical or physical transformations before
%testing can begin (e.g., programs must be transferred from tape to disk).
%Supply references to the following test item documentation, if it exists:
%a) Requirements specification;
%b) Design specification;
%c) Users guide;
%d) Operations guide;
%e) Installation guide.
%Reference any incident reports relating to the test items.
%Items that are to be specifically excluded from testing may be identified.

We use with the 0.6 version of the SAGA-A C++
implementation built on the 1.34  boost library. This version is 
installed on a linux x86\_64 machine running debian etch distribution.
 
\begin{itemize}
 \item  Boost 1.34 C++ library was downloaded from the web site:\\
 	\mbox{\href{http://www.boost.org/}{http://www.boost.org/}}. 
 \item SAGA-A 0.6 C++ implementation library was directly checked out 
   from the SVN repository (revision 768):\\
	\mbox{\href{https://svn.cct.lsu.edu/repos/saga/branches/sage-0.6/trunk }
	{https://svn.cct.lsu.edu/repos/saga/branches/sage-0.6/trunk }}\\
   The 10-day old SAGA-A 0.7 version released on November $12^{th}$ was also downloaded
   but abandoned because the examples did not compile correctly and
       revealed a mismatch of header files. The same problem occurred for
       our test application. The SAGA development team recommended us to
       take the previous release instead.
\end{itemize}

\subsubsection{Features to be Tested}
%Identify all software features and combinations of software features to be tested. Identify the test design
%specification associated with each feature and each combination of features.
In this document, the following features will be tested:
\begin{itemize}
 \item Submission, monitoring, deletion of jobs.
 \item Creation, deletion, copying and renaming of files.
\end{itemize}


The SAGA filesystem API entirely covers the POSIX API and makes it
usable to several grid platforms. A user may then prefer to use
the SAGA API instead of regular POSIX file API, in order to obtain a
portable application over these environment


Note that the C++ implementation of SAGA specification
is still ongoing and the tested release (0.6) partially supports 
it. 

\subsubsection{Features not to be Tested}
%Identify all features and significant combinations of features that will not be tested and the reasons.
The following features are not tested as they are not supported by
the 0.6 release:
\begin{itemize}
 \item Remote procedure Calls.
 \item Stream management.
 \item  Replica management
\end{itemize}
As mentioned in the SAGA specification, these features are part
of the complete implementation of SAGA which should be available on XtreemOS.

\subsubsection{Approach}

%Describe the overall approach to testing. For each major group of features or feature combinations, specify
%the approach that will ensure that these feature groups are adequately tested. Specify the major activities,
%techniques, and tools that are used to test the designated groups of features.
%The approach should be described in sufficient detail to permit identification of the major testing tasks and
%estimation of the time required to do each one.
%Specify the minimum degree of comprehensiveness desired. Identify the techniques that will be used to
%judge the comprehensiveness of the testing effort (e.g., determining which statements have been executed at
%least once). Specify any additional completion criteria (e.g., error frequency). The techniques to be used to
%trace requirements should be specified.
%Identify significant constraints on testing such as test item availability, testing resource availability, and
%deadlines.
The purpose of the early test is to provide feedback to developers
about the implemented features and about the fulfillment of
requirements.  

Tests are performed by installing a version of SAGA that runs locally on
one testbed machine. That way, we could test calls to the SAGA C++ API
but all the spawned job were executing locally. 

%For features not yet available while needed by EDF Applications, SAGA
%specification is checked to be certain that final XtreemOS
%implementation of SAGA will fulfill application requirements.



\subsubsection{Item Pass Criteria}
%Specify the criteria to be used to determine whether each test item has
%passed or failed testing.

SAGA will pass a test if:
\begin{itemize}
\item running application under the condition defined by the test design
      specification does not crash the application.
\item an application calling SAGA, under the condition of the test,
      gives the same results as if jobs or files were managed directly
      from a shell prompt.

\end{itemize}

%\subsubsection{Suspension Criteria and Resumption Requirements}
%Specify the criteria used to suspend all or a portion of the testing activity on the test items associated with
%this plan. Specify the testing activities that must be repeated, when testing is resumed.

\subsubsection{Testing Tasks}
%Identify the set of tasks necessary to prepare for and perform testing. Identify all intertask dependencies and
%any special skills required.

The testing tasks include:
\begin{itemize}
\item building and installing SAGA
\item running simple programs that submit, monitor, and delete jobs
      running the ZEPHYR application.
\item running simple programs managing files (initialization of a dir
      tree, creation, writing, reading, deletion and renaming of files).
\end{itemize}

The testing tasks does not yet include:

\begin{itemize}
 \item running and benchmarking SAGA asynchronous tasks support.
 \item performing remote procedure calls.
 \item sharing streams among SAGA processes
 \item evaluating the Advert Service
\end{itemize}

As mentioned in the SAGA specification, these features are part
of the complete implementation of SAGA which should be available on XtreemOS.
These latter features are planned to be tested in succeeding WP4.2 deliverables.

\subsubsection{Environmental Needs}
%Specify both the necessary and desired properties of the test environment. This specification should contain
%the physical characteristics of the facilities including the hardware, the communications and system software,
%the mode of usage (e.g., stand-alone), and any other software or supplies needed to support the test.
%Also specify the level of security that must be provided for the test facilities, system software, and proprietary
%components such as software, data, and hardware.
%Identify special test tools needed. Identify any other testing needs (e.g., publications or office space). Identify
%the source for all needs that are not currently available to the test group.

\label{api:env}

SAGA and Boost have no special hardware requirements. They could be
executed on a regular desktop two-processor PC with 2 GB RAM, and
2 GHz CPU clock speed.  They are designed to be independent of any
specific Linux distribution. On the testbed, we run Linux Kernel
2.6.18-4-amd64 with SMP and 64 bit support. SAGA and Boost have been
compiled and installed with the pre-installed third-party software and
libraries :
\begin{itemize}
 \item gmake 3.81
 \item gcc 4.1.2
 \item autoconf 2.61
 \item automake 1.9.6
\end{itemize}



\subsubsection{Responsibilities}
%Identify the groups responsible for managing, designing, preparing, executing, witnessing, checking, and
%resolving. In addition, identify the groups responsible for providing the test items identified in Section 'Test Items'  and the
%environmental needs identified in Section 'Environmental needs'.
%These groups may include the developers, testers, operations staff, user representatives, technical support
%staff, data administration staff, and quality support staff.

This test plan and the included tests are the
responsibility of EDF.


\subsubsection{Schedule}
%Include test milestones identified in the software project schedule as well as all item transmittal events.
%Define any additional test milestones needed. Estimate the time required to do each testing task. Specify the
%schedule for each testing task and test milestone. For each testing resource (i.e., facilities, tools, and staff),
%specify its periods of use.
Tests have been performed from November $15^{th}$ to December
$14^{th}$ 2007.

\subsubsection{Risks and Contingencies}
%Identify the high-risk assumptions of the test plan. Specify contingency plans for each (e.g., delayed delivery
%of test items might require increased night shift scheduling to meet the delivery date).

Delays in testing caused by unexpected installation and
execution problems are envisaged.% as well as search for
%features without knowing they have not been
%implemented yet in the tested release. 
