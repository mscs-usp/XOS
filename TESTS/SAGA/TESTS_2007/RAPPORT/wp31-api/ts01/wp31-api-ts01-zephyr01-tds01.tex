%%%%%%%%%%%%%%%%%%%%%%%%%%%%%%%%%%%%%%%%%
% Template for describing the test design specification.
% structure and annotations are quoted from IEEE Std 829-1998
%%%%%%%%%%%%%%%%%%%%%%%%%%%%%%%%%%%%%%%%%


%\subsection{Test Design Specification: Available features of SAGA API on
%  a regular Linux Box}
\starttds{wp31-api-ts01-zephyr01-tds01}
%Specify the unique identifier assigned to this test design specification. 
%\noindent\textbf{Test design specification identifier: wp31-api-ts01-zephyr01-tds01}%insert file name here without extension '.tex'

%\noindent\textbf{Test plan reference: wp31-api-ts01-tp} %Supply a reference (identifier) to the associated test plan.

%Purpose
%To specify refinements of the test approach and to identify the features to be tested by this design and its
%associated tests.



%Outline:
%A test design specification shall have the following structure:
%a) Test design specification identifier;
%b) Features to be tested;
%c) Approach refinements;
%d) Test identification;
%e) Feature pass/fail criteria.
%The sections shall be ordered in the specified sequence. Additional sections may be included at the end. If
%some or all of the content of a section is in another document, then a reference to that material may be listed
%in place of the corresponding content. The referenced material must be attached to the test design specification
%or available to users of the design specification.
%Details on the content of each section are contained in the following subclauses.

\subsubsection{Features to be Tested}
%Identify the test items and describe the features and combinations of features that are the object of this
%design specification. Other features may be exercised, but need not be identified.
%For each feature or feature combination, a reference to its associated requirements in the item requirement
%specification or design description should be included.

In this release, we test some of the available features of the SAGA C++
API running on a regular Linux system. In particular, the support of
\begin{itemize}
 \item Job Management : spawn, monitor, cancel or terminate a job
 \item File management : create, copy, rename, delete files in different
       directories.
\end{itemize}

\subsubsection{Approach Refinements}
%Specify refinements to the approach described in the test plan. Include specific test techniques to be used.
%The method of analyzing test results should be identified (e.g., comparator programs or visual inspection).
%Specify the results of any analysis that provides a rationale for test case selection. For example, one might
%specify conditions that permit a determination of error tolerance (e.g., those conditions that distinguish valid
%inputs from invalid inputs).
%Summarize the common attributes of any test cases. This may include input constraints that must be true for
%every input in the set of associated test cases, any shared environmental needs, any shared special procedural
%requirements, and any shared case dependencies.

After compiling and installing SAGA (first test case), we test the
SAGA implementation and API in two ways: 
\begin{itemize}
 \item a test application that runs, monitors, and cancels a ZEPHYR application.
 \item a test program that performs file operations (creation of a directory
       tree, creation, copy, renaming or deletion of files)
\end{itemize}

\subsubsection{Test Identification}
%List the identifier and a brief description of each test case associated with this design. A particular test case
%may be identified in more than one test design specification. List the identifier and a brief description of each
%procedure associated with this test design specification.

Two test cases relate to this design specification :
\begin{itemize}
 \item {\refexpdoc{wp31-api-ts01-zephyr01-tcs01}} dealing with the
       installation of SAGA and focusing on job management issues.
 \item {\refexpdoc{wp31-api-ts01-zephyr01-tcs01}} dealing with files
       handling operations.
\end{itemize}

\subsubsection{Feature Pass/Fail Criteria}
%Specify the criteria to be used to determine whether the feature or feature combination has passed or failed.

Tests are considered as passed when the test program behavior and output
correspond to the SAGA specification.
