%%%%%%%%%%%%%%%%%%%%%%%%%%%%%%%%%%%%%%%%%
% Template for describing the test summary report.
% structure and annotations are quoted from IEEE Std 829-1998
%%%%%%%%%%%%%%%%%%%%%%%%%%%%%%%%%%%%%%%%%


%\subsection{Test Summary Report}
\starttsr{wp31-api-ts01-tsr}{Evaluation of the SAGA API}
%Specify the unique identifier assigned to this test summary report.
%\noindent\textbf{Test summary report identifier: wp31-api-ts01-tsr}%insert file name here without extension '.tex'

%Purpose:
%To summarize the results of the designated testing activities and to provide evaluations based on these
%results.

Although the SAGA API component
has not been released yet for XtreemOS, this first
set of tests gives a first impression of the job and file management interfaces expected.

% good idea of what will be the XtreemOS way to manage jobs and files.



%Outline:
%A test summary report shall have the following structure:
%a) Test summary report identifier;
%b) Summary;
%c) Variances;
%d) Comprehensive assessment;
%e) Summary of results;
%f) Evaluation;
%g) Summary of activities;
%h) Approvals. Bernd: we do not use this
%The sections shall be ordered in the specified sequence. Additional sections may be included just prior to
%Approvals. If some or all of the content of a section is in another document, then a reference to that material
%may be listed in place of the corresponding content. The referenced material must be attached to the test
%summary report or available to users of the summary report.
%Details on the content of each section are contained in the following subclauses.


\subsubsection{Summary}
%Summarize the evaluation of the test items. Identify the items tested, indicating their version/revision level.
%Indicate the environment in which the testing activities took place.
%For each test item, supply references to the following documents if they exist: test plan, test design specifications,
%test procedure specifications, test item transmittal reports, test logs, and test incident reports.

The 0.6  release of the SAGA-A C++ API
has been evaluated 
\begin{itemize}
 \item with a test application 
     managing jobs running the  EDF application Zephyr.
 \item with a simple program managing 
files. It is expected that the EDF applications ZEPHYR, MODERATO or SIMEON can be suitably adapted to the SAGA file management interface in order to address files on
XtreemOS.
\end{itemize}

%As well the document ``A simple API for Grid Application, T. Goodale et
%all., GWD-R.90 SAGA-CORE-WG'' have been carefully read in its 1.0 RC.6 version.


%\subsubsection{Variances}
%Report any variances of the test items from their design specifications. Indicate any variances from the test
%plan, test designs, or test procedures. Specify the reason for each variance.

At this time, the tests could only cover job and file management and just
have an informative value because the SAGA for XtreemOS has not been released 
yet. However, the installation of SAGA and the execution of the test programs
have allowed us to gain first insights into the concepts.

During the installation of SAGA and Boost using autotools, some
minor operations have to be done manually (see \refexpdoc{wp31-api-ts01-zephyr01-tps01}):
\begin{itemize}
 \item renaming of a boost shared library
 \item copy of saga library in the
       library path not mentioned in documentation
\end{itemize} 

The need for these manual operations is not well-identified
yet and may be due to the environment of the test machine.
This issue will be further investigated when testing the next
release of SAGA.


A problem was also forwarded to the SAGA team concerning a wrong status
reported for an ended job. This referred to an already identified
bug in SAGA.


\subsubsection{Comprehensiveness Assessment}
%Evaluate the comprehensiveness of the testing process against the comprehensiveness criteria specified in
%the test plan (section 'Approach' in test plan document) if the plan exists. Identify features or feature combinations that were not sufficiently
%tested and explain the reasons.

The testing process was in line with the approach given in \refexpdoc{wp31-api-ts01-tp}.


%\subsubsection{Summary of Results}
%%Summarize the results of testing. Identify all resolved incidents and summarize their resolutions. Identify all
%%unresolved incidents.

%After this first partial evaluation of SAGA C++ API (release 0.6), one arrives at the following conclusions:
%\begin{itemize}
%\item first tests of the SAGA API on a Linux system do not allow to give a final assessment of the fulfillment of requirements  \ref{req:other_apis_as_basis},
%      \ref{req:posix}. However, the first impression is that the SAGA development is prepared for the integration with XtreemOS services.  
% \item The first contact with the C++ API was satisfying. C and Java
%       wrapping are expected to fulfill requirement \ref{req:api_language}.
%\end{itemiz


\subsubsection{Evaluation}
%Provide an overall evaluation of each test item including its limitations. This evaluation shall be based upon
%the test results and the item level pass/fail criteria. An estimate of failure risk may be included.

% Bernd: This evaluation should also state which WP4.2 requirements passed/failed. 


Below we present a list of XtreemOS-related requirements and their fulfillment status. 
We also list requirements that are not implemented yet, and therefore had not been tested.
% However, these requirements must be fulfilled in future versions of XtreemFS.

\begin{itemize}
\item \ref{req:other_apis_as_basis} not fulfilled, since not
      implemented yet, but SAGA also supports GridFTP. Note that GAT support
      is currently  implemented in the 0.7 version of SAGA and that
      the job management SAGA API reflects the DRMAA API very closely.
\item \ref{req:posix} not fulfilled, since not
      implemented yet. Still the SAGA specification makes several
      references to POSIX. The full POSIX compliance still needs to be evaluated.
\item \ref{req:api_language} not fulfilled, since not
      implemented yet, but a C++ API is already available for the execution on a single Linux machine.
\item \ref{req:degree_of_interoperability} not fulfilled, since not
      implemented yet.
% \item \ref{req:meta_access_gran} not tested, since access right management is not fully implemented by XtreemFS
% \item \ref{req:monitoring} not tested, since not implemented
% \item \ref{req:data_access_time} ???
% \item \ref{req:replica_limitation} not tested, since not implemented
% \item \ref{req:data_versioning} not tested, since not implemented
% \item \ref{req:data_transmission_monitoring} not tested, since not implemented
% \item \ref{req:GOM} partly fulfilled, only the basic functionality could be tested, transactions are scheduled for later testing (wp34-xfs-ts01-wiss01-tds02)
% \item \ref{req:posix} partly fulfilled
% \item \ref{req:data_and_comput_intensive} not fulfilled, scalability and performance of XtreemFS data management were rather poor
% \item \ref{req:terabyte_storage} not tested, because of lack of hardware resources
% \item \ref{req:sec_acl} not fulfilled and not fully tested, since ACLs are not yet supported in XtreemFS
\end{itemize}


In the next release of this document, more extended tests will be
performed hopefully on the integrated version of XtreemOS.

%\subsubsection{Summary of Activities}
%Summarize the major testing activities and events. Summarize resource consumption data, e.g., total staffing
%level, total machine time, and total elapsed time used for each of the major testing activities.
