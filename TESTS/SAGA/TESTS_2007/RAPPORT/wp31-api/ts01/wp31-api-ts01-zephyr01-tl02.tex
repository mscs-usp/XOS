%%%%%%%%%%%%%%%%%%%%%%%%%%%%%%%%%%%%%%%%%
% Template for describing the test log.
% structure and annotations are quoted from IEEE Std 829-1998
%%%%%%%%%%%%%%%%%%%%%%%%%%%%%%%%%%%%%%%%%

%\subsection{Test log: File Management}
\starttl{wp31-api-ts01-zephyr01-tl02}
%Specify the unique identifier assigned to this test log.
%\noindent\textbf{Test log identifier: wp31-api-ts01-zephyr01-tl02}%insert file name here without extension '.tex'



%Purpose:
%To provide a chronological record of relevant details about the execution of tests.

%Outline:
%A test log shall have the following structure:
%a) Test log identifier;
%b) Description;
%c) Activity and event entries.
%The sections shall be ordered in the specified sequence. Additional sections may be included at the end. If
%some or all of the content of a section is in another document, then a reference to that material may be listed
%in place of the corresponding content. The referenced material must be attached to the test log or available to
%users of the log.
%Details on the content of each section are contained in the following subclauses.


\subsubsection{Description}
%Information that applies to all entries in the log except as specifically noted in a log entry should be included
%here. The following information should be considered:

%a) Identify the items being tested including their version/revision levels. For each of these items, supply
%a reference to its transmittal report, if it exists.
%b) Identify the attributes of the environments in which the testing is conducted. Include facility identifi-
%cation, hardware being used (e.g., amount of memory being used, CPU model number, and number
%and model of tape drives, and/or mass storage devices), system software used, and resources available
%(e.g., the amount of memory available).

This test has been done in the hardware and software environment
described in \refexpdoc{api:env} 

It consists in using SAGA API to handle files (creation, renaming, copy,
or deletion) in a directory tree.




\subsubsection{Activity and Event Entries}
%For each event, including the beginning and end of activities, record the occurrence date and time along with
%the identity of the author.
%The information in (1) through (5) should be considered:

%(1) 
%\paragraph{Execution Description}
%Record the identifier of the test procedure being executed and supply a reference to its specification. Record
%all personnel present during the execution including testers, operators, and observers. Also indicate the function
%of each individual.

The tests described here have been conducted by Samuel Kortas (EDF). The
execution procedure is described in specification
\refexpdoc{wp31-api-ts01-zephyr01-tps02}.

%(2)
%\paragraph{Procedure Results}
%For each execution, record the visually observable results (e.g., error messages generated, aborts, and
%requests for operator action). Also record the location of any output (e.g., reel number). Record the successful
%or unsuccessful execution of the test.


\subsubsection{Execution of the test program}

Calling the test program several times, after a ``touch es.txt'',
the following correct
 output is always produced:

\begin{small}
\begin{lstlisting}
 cleaning Working directory....
 creating directory tree....
/tmp/SAGA:
total 4
drwxr-xr-x 3 kortas edf 4096 2007-12-13 10:23 sup1k

/tmp/SAGA/sup1k:
total 4
drwxr-xr-x 3 kortas edf 4096 2007-12-13 10:23 sup1M

/tmp/SAGA/sup1k/sup1M:
total 4
drwxr-xr-x 2 kortas edf 4096 2007-12-13 10:23 sup5M

/tmp/SAGA/sup1k/sup1M/sup5M:
total 0

 creating files....
/tmp/SAGA:
total 8
-rw-r--r-- 1 kortas edf    1 2007-12-13 10:23 es.txt
drwxr-xr-x 3 kortas edf 4096 2007-12-13 10:23 sup1k

/tmp/SAGA/sup1k:
total 8
-rw-r--r-- 1 kortas edf    1 2007-12-13 10:23 es1k.txt
drwxr-xr-x 3 kortas edf 4096 2007-12-13 10:23 sup1M

/tmp/SAGA/sup1k/sup1M:
total 8
-rw-r--r-- 1 kortas edf 1000 2007-12-13 10:23 es1M.txt
drwxr-xr-x 2 kortas edf 4096 2007-12-13 10:23 sup5M

/tmp/SAGA/sup1k/sup1M/sup5M:
total 112
-rw-r--r-- 1 kortas edf 100000 2007-12-13 10:23 es100M.txt
-rw-r--r-- 1 kortas edf   5000 2007-12-13 10:23 es5M.txt

 copying file SAGA/es.txt to SAGA/as.txt....
/tmp/SAGA:
total 12
-rw-r--r-- 1 kortas edf    1 2007-12-13 10:23 as.txt
-rw-r--r-- 1 kortas edf    1 2007-12-13 10:23 es.txt
drwxr-xr-x 3 kortas edf 4096 2007-12-13 10:23 sup1k

/tmp/SAGA/sup1k:
total 8
-rw-r--r-- 1 kortas edf    1 2007-12-13 10:23 es1k.txt
drwxr-xr-x 3 kortas edf 4096 2007-12-13 10:23 sup1M

/tmp/SAGA/sup1k/sup1M:
total 8
-rw-r--r-- 1 kortas edf 1000 2007-12-13 10:23 es1M.txt
drwxr-xr-x 2 kortas edf 4096 2007-12-13 10:23 sup5M

/tmp/SAGA/sup1k/sup1M/sup5M:
total 112
-rw-r--r-- 1 kortas edf 100000 2007-12-13 10:23 es100M.txt
-rw-r--r-- 1 kortas edf   5000 2007-12-13 10:23 es5M.txt

 renaming file SAGA/es.txt in SAGA/es2.txt....
/tmp/SAGA:
total 12
-rw-r--r-- 1 kortas edf    1 2007-12-13 10:23 as.txt
-rw-r--r-- 1 kortas edf    1 2007-12-13 10:23 es2.txt
drwxr-xr-x 3 kortas edf 4096 2007-12-13 10:23 sup1k

/tmp/SAGA/sup1k:
total 8
-rw-r--r-- 1 kortas edf    1 2007-12-13 10:23 es1k.txt
drwxr-xr-x 3 kortas edf 4096 2007-12-13 10:23 sup1M

/tmp/SAGA/sup1k/sup1M:
total 8
-rw-r--r-- 1 kortas edf 1000 2007-12-13 10:23 es1M.txt
drwxr-xr-x 2 kortas edf 4096 2007-12-13 10:23 sup5M

/tmp/SAGA/sup1k/sup1M/sup5M:
total 112
-rw-r--r-- 1 kortas edf 100000 2007-12-13 10:23 es100M.txt
-rw-r--r-- 1 kortas edf   5000 2007-12-13 10:23 es5M.txt

 deleting file SAGA/as.txt ....
/tmp/SAGA:
total 8
-rw-r--r-- 1 kortas edf    1 2007-12-13 10:23 es2.txt
drwxr-xr-x 3 kortas edf 4096 2007-12-13 10:23 sup1k

/tmp/SAGA/sup1k:
total 8
-rw-r--r-- 1 kortas edf    1 2007-12-13 10:23 es1k.txt
drwxr-xr-x 3 kortas edf 4096 2007-12-13 10:23 sup1M

/tmp/SAGA/sup1k/sup1M:
total 8
-rw-r--r-- 1 kortas edf 1000 2007-12-13 10:23 es1M.txt
drwxr-xr-x 2 kortas edf 4096 2007-12-13 10:23 sup5M

/tmp/SAGA/sup1k/sup1M/sup5M:
total 112
-rw-r--r-- 1 kortas edf 100000 2007-12-13 10:23 es100M.txt
-rw-r--r-- 1 kortas edf   5000 2007-12-13 10:23 es5M.txt

real	0m11.977s
user	0m6.108s
sys	0m5.256s
\end{lstlisting}
 \end{small}



We checked that the file produced (as.txt) was the copy of the 
initial file (es.txt). Results obtained are the same as if
the equivalent shell script \refexpdoc{api:file_script} is run.


The observed time is always around 12 seconds to run the C++ test
application on this system.

Summary: We consider the test as passed.
Using Saga, one could easily create, rename, delete a file.

%(3)
%\paragraph{Environmental Information}
%Record any environmental conditions specific to this entry (e.g., hardware substitutions).

%(4)
%\subsubsection{Anomalous Events}
%Record what happened before and after an unexpected event occurred (e.g., A summary display was
%requested and the correct screen displayed, but response seemed unusually long. A repetition produced the
%same prolonged response). Record circumstances surrounding the inability to begin execution of a test
%procedure or failure to complete a test procedure (e.g., a power failure or system software problem).






%(5)
%\subsubsection{Incident Report Identifiers}
%Record the identifier of each test incident report, whenever one is generated.

