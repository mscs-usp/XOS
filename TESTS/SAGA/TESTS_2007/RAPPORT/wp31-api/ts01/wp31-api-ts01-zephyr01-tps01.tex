%%%%%%%%%%%%%%%%%%%%%%%%%%%%%%%%%%%%%%%%%
% Template for describing the test procedure specification.
% structure and annotations are quoted from IEEE Std 829-1998
%%%%%%%%%%%%%%%%%%%%%%%%%%%%%%%%%%%%%%%%%



%\subsection{Test Procedure Specification: Job Management}
\starttps{wp31-api-ts01-zephyr01-tps01}
%Specify the unique identifier assigned to this test procedure specification.
%\noindent\textbf{Test log identifier: wp31-api-ts01-zephyr01-tl01} %insert file name here without extension '.tex'

%\noindent\textbf{Test design specification reference: wp31-api-ts01-zephyr01-tds01} %Supply a reference (identifier) to the associated test design specification.

%Purpose:
%To specify the steps for executing a set of test cases or, more generally, the steps used to analyze a software
%item in order to evaluate a set of features.

%Outline:
%A test procedure specification shall have the following structure:
%a) Test procedure specification identifier.
%b) Purpose;
%c) Special requirements;
%d) Procedure steps.
%The sections shall be ordered in the specified sequence. Additional sections, if required, may be included at
%the end. If some or all of the content of a section is in another document, then a reference to that material
%may be listed in place of the corresponding content. The referenced material must be attached to the test
%procedure specification or available to users of the procedure specification.
%Details on the content of each section are contained in the following subclauses.


\subsubsection{Purpose}
%Describe the purpose of this procedure. If this procedure executes any test cases, provide a reference for each
%of them.
%In addition, provide references to relevant sections of the test item documentation (e.g., references to usage
%procedures).

This procedure executes the test case \refexpdoc{wp31-api-ts01-zephyr01-tcs01}.

%\subsubsection{Special Requirements}
%Identify any special requirements that are necessary for the execution of this procedure. These may include
%prerequisite procedures, special skills requirements, and special environmental requirements.


\subsubsection{Procedure Steps}
%Include the following steps as applicable.

%\paragraph{Log}
%Describe any special methods or formats for logging the results of test execution, the incidents observed, and
%any other events pertinent to the test (see Clauses 9 and 10).

\paragraph{Set Up}
%Describe the sequence of actions necessary to prepare for execution of the procedure.

The generic test procedure setup includes:
\begin{itemize}
\item Installation and compilation of SAGA.
\item Compiling and running the test program
\end{itemize}

Installation and compilation of SAGA on Debian Linux etch
distribution including the following steps:
\begin{enumerate}
 \item Download of Boost 1.34.1 
 \item Compilation of Boost and manual linking of  a shared library.
 \item Checking out the 0.6 release of SAGA.
 \item Compilation of SAGA.
\end{enumerate}



\paragraph{Start}
%Describe the actions necessary to begin execution of the procedure.

In order to start the test one needs to perform the following steps:
\begin{enumerate}
\item Download, compile and install the 1.34.1 boost C++ library :
\begin{lstlisting}
 Download boost_1_34_1.tar.gz  from website http://www.boost.org	
 cd /home/kortas/BUILD
 tar xvfz /home/kortas/XTREEMOS/WP4.2/SAGA/boost_1_34_1.tar.gz 
 cd boost_1_34_1/
 ./configure --prefix=/home/kortas/SOFTS/boost_34_1
 make
 make install
\end{lstlisting}
\item Download, compile and install the 0.6 SAGA-A C++ implementation
\begin{lstlisting}
 cd /home/kortas/BUILD
 svn checkout svn co https://svn.cct.lsu.edu/repos/saga/branches/saga-0.6/trunk
 mv trunk saga-c++-0.6-src
 export LD_LIBRARY_PATH=/home/kortas/SOFTS/boost_34_1/lib:$LD_LIBRARY_PATH
 ./configure --with-boost=/home/kortas/SOFTS/boost_34_1 --prefix=/home/kortas/SOFTS/saga-06
 make
\end{lstlisting}
As stated in the release notes, but not on the website, the gcc
      version used must be 3.4.8 or higher otherwise the
      code will not compile because of inconsistent C++ headers.

\item Compile the following  test program inspired by the one in \\
    /home/kortas/BUILD/saga-c++-0.6/examples/packages/job/job\_run.cpp
      \\
     and running a Zephyr application.
\begin{lstlisting}
//  Copyright (c) 2005-2007 Andre Merzky (andre@merzky.net)
// 
//  Distributed under the Boost Software License, Version 1.0. (See accompanying 
//  file LICENSE_1_0.txt or copy at http://www.boost.org/LICENSE_1_0.txt)

#include <iostream>
#include <saga.hpp>

std::string state_to_string (saga::job::state state)
{
  switch ( state )
  {
    case saga::job::New       : return ("New      ");
    case saga::job::Running   : return ("Running  ");
    case saga::job::Suspended : return ("Suspended");
    case saga::job::Done      : return ("Done     ");
    case saga::job::Failed    : return ("Failed   ");
    case saga::job::Canceled  : return ("Canceled ");
    default                   : 
        std::cout << "Unknown state: " << state << std::endl;
        return ("Unknown  ");
  }
}



/////////////
int main (int argc, char* argv[])
{
    namespace attribs = saga::attributes;
    
    saga::job_description jd_valid;
    jd_valid.set_attribute (attribs::job_description_executable,  "/home/kortas/EX/ZEPHYR/SRC/zephyr");
    
    saga::job_service s1("any://localhost");
    
    saga::job j1 = s1.create_job(jd_valid);
    saga::job j2 = s1.create_job(jd_valid);

    std::cout << "NOT RUN! Job state is: " << state_to_string(j1.get_state()) << std::endl;

    j1.run();
    std::cout << "JobID: " << j1.get_job_id() << std::endl;
    std::cout << "Just RUN! Job state is: " << state_to_string(j1.get_state()) << std::endl;

    j1.suspend();
    std::cout << "SUSPENDED! Job state is: " << state_to_string(j1.get_state()) << std::endl;
    
    j1.resume();
    std::cout << "RESUMED! Job state is: " << state_to_string(j1.get_state()) << std::endl;
        
    
    std::cout << "NOT DONE! Job state is: " << state_to_string(j1.get_state()) << std::endl;
    
    j1.wait();

    std::cout << "AWAITED! Job state is: " << state_to_string(j1.get_state()) << std::endl;
    
    sleep(30);
    std::cout << "should be DONE by now! Job state is: " << state_to_string(j1.get_state()) << std::endl;

    //j1.cancel();
    


    j2.run();
    std::cout << "JobID: " << j2.get_job_id() << std::endl;
    std::cout << "Just RUN! Job state is: " << state_to_string(j2.get_state()) << std::endl;

    j2.cancel();
    std::cout << "CANCELED! Job state is: " << state_to_string(j2.get_state()) << std::endl;

    return 0;
}






\end{lstlisting}

using the following makefile
\begin{lstlisting}
#  Copyright (c) 2005-2006 Andre Merzky (andre@merzky.net)
SAGA_ROOT=/home/kortas/SOFTS/BUILD/saga-c++-0.6-src

SAGA_SRC         = main.cpp
SAGA_BIN         = main

SAGA_ADD_BIN_OBJ = $(SAGA_SRC:%.cpp=%.o)

include $(SAGA_ROOT)/make/saga.application.mk
\end{lstlisting}

 \item Finally run the test program
\begin{lstlisting}
 export LD_LIBRARY_PATH=/home/kortas/SOFTS/saga-06/lib:\
	/home/kortas/SOFTS/boost_34_1/lib:$LD_LIBRARY_PATH:
 export SAGA_LOCATION=/home/kortas/SOFTS/saga-06/
 ./main
\end{lstlisting}
\end{enumerate}


%\paragraph{Proceed}
%Describe any actions necessary during execution of the procedure.

%\paragraph{Measure}
%Describe how the test measurements will be made (e.g., describe how remote terminal response time is to be
%measured using a network simulator).

%\paragraph{Shut Down}
%Describe the actions necessary to suspend testing, when unscheduled events dictate.

%\paragraph{Restart}
%Identify any procedural restart points and describe the actions necessary to restart the procedure at each of
%these points.

%\paragraph{Stop}
%Describe the actions necessary to bring execution to an orderly halt.


%\paragraph{Wrap Up}
%Describe the actions necessary to restore the environment.


\paragraph{Contingencies}
%Describe the actions necessary to deal with anomalous events that may occur during execution.
\label{api:contingencies}
\begin{itemize}
 \item After  compiling Boost, we also need to manually add an additional link,
      otherwise SAGA would not link properly with boost searching the
      library libboost\_serialization-gcc41-mt-d-1\_34\_1.so that
      did not exist.
 \item In order to run properly the example bash\_correction in \\
       /home/kortas/SOFTS/BUILD/saga-c++-0.6-src/example, we need to
       compile the package logical\_file in \\
       /home/kortas/SOFTS/BUILD/0.6/examples/packages/logical\_file:
       and copy by hand the \\
       libsaga\_package\_logicalfile.so into
       the /home/kortas/SOFTS/saga-06/lib directory.
\end{itemize}

