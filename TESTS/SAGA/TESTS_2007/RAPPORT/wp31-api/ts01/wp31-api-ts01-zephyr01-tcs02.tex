%%%%%%%%%%%%%%%%%%%%%%%%%%%%%%%%%%%%%%%%%
% Template for describing the test case specification.
% structure and annotations are quoted from IEEE Std 829-1998
%%%%%%%%%%%%%%%%%%%%%%%%%%%%%%%%%%%%%%%%%


%\subsection{Test Case Specification: File Management}
\starttcs{wp31-api-ts01-zephyr01-tcs02}{File Management}
%\noindent\textbf{Test case specification identifier: wp31-api-ts01-zephyr01-tcs02}

%Purpose:
%To define a test case identified by a test design specification.

%Outline:
%A test case specification shall have the following structure:
%a) Test case specification identifier;
%b) Test items;
%c) Input specifications;
%d) Output specifications;
%e) Environmental needs;
%f) Special procedural requirements;
%g) Intercase dependencies.
%The sections shall be ordered in the specified sequence. Additional sections may be included at the end. If
%some or all of the content of a section is in another document, then a reference to that material may be listed
%in place of the corresponding content. The referenced material must be attached to the test case specification
%or available to users of the case specification.
%Since a test case may be referenced by several test design specifications used by different groups over a long
%time period, enough specific information must be included in the test case specification to permit reuse.
%Details on the content of each section are contained in the following subclauses.


\subsubsection{Test Items}
%Identify and brießy describe the items and features to be exercised by this test case.
%For each item, consider supplying references to the following test item documentation:
%a) Requirements specification;
%b) Design specification;
%c) Users guide;
%d) Operations guide;
%e) Installation guide.

We will test the behavior of the SAGA filesystem API by creating,
renaming, copying and deleting files of different sizes in a set of directories on the
target machine using the saga API. 



\subsubsection{Input Specifications}
%Specify each input required to execute the test case. Some of the inputs will be specified by value (with
%tolerances where appropriate), while others, such as constant tables or transaction files, will be specified by
%name. Identify all appropriate databases, files, terminal messages, memory resident areas, and values passed
%by the operating system.
%Specify all required relationships between inputs (e.g., timing).

Input is generated by the test program itself. No additional input is required.

\subsubsection{Output Specifications}
%Specify all of the outputs and features (e.g., response time) required of the test items. Provide the exact value
%(with tolerances where appropriate) for each required output or feature.

The test program prints its output on the
standard output stream reproduced in the following of this document. 

\subsubsection{Environmental Needs}

%\paragraph{Hardware}
%Specify the characteristics and configurations of the hardware required to execute this test case (e.g.,
%132 character 24 line CRT).

%TODO


\paragraph{Software}
%Specify the system and application software required to execute this test case. This may include system software
%such as operating systems, compilers, simulators, and test tools. In addition, the test item may interact
%with application software.

A simple program has been used as a test program. The hardware and software
testing environment is the one described in \refexpdoc{api:env}


%\paragraph{Other}
%Specify any other requirements such as unique facility needs or specially trained personnel.

%\subsubsection{Special Procedural Requirements}
%Describe any special constraints on the test procedures that execute this test case. These constraints may
%involve special set up, operator intervention, output determination procedures, and special wrap up.

%\subsubsection{Intercase Dependencies}
%List the identifiers of test cases that must be executed prior to this test case. Summarize the nature of the
%dependencies.
