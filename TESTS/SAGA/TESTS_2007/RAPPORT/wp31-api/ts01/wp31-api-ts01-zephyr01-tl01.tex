%%%%%%%%%%%%%%%%%%%%%%%%%%%%%%%%%%%%%%%%%
% Template for describing the test log.
% structure and annotations are quoted from IEEE Std 829-1998
%%%%%%%%%%%%%%%%%%%%%%%%%%%%%%%%%%%%%%%%%

%\subsection{Test log: Job Management}
\starttl{wp31-api-ts01-zephyr01-tl01}
%Specify the unique identifier assigned to this test log.
%\noindent\textbf{Test log identifier: wp31-api-ts01-zephyr01-tl01}%insert file name here without extension '.tex'



%Purpose:
%To provide a chronological record of relevant details about the execution of tests.

%Outline:
%A test log shall have the following structure:
%a) Test log identifier;
%b) Description;
%c) Activity and event entries.
%The sections shall be ordered in the specified sequence. Additional sections may be included at the end. If
%some or all of the content of a section is in another document, then a reference to that material may be listed
%in place of the corresponding content. The referenced material must be attached to the test log or available to
%users of the log.
%Details on the content of each section are contained in the following subclauses.


\subsubsection{Description}
%Information that applies to all entries in the log except as specifically noted in a log entry should be included
%here. The following information should be considered:
%a) Identify the items being tested including their version/revision levels. For each of these items, supply
%a reference to its transmittal report, if it exists.
%b) Identify the attributes of the environments in which the testing is conducted. Include facility identifi-
%cation, hardware being used (e.g., amount of memory being used, CPU model number, and number
%and model of tape drives, and/or mass storage devices), system software used, and resources available
%(e.g., the amount of memory available).

The tests have been done in the hardware and software environment
described in \refexpdoc{api:env} 




\subsubsection{Activity and Event Entries}
%For each event, including the beginning and end of activities, record the occurrence date and time along with
%the identity of the author.
%The information in (1) through (5) should be considered:

%(1) 
\paragraph{Execution Description}
%Record the identifier of the test procedure being executed and supply a reference to its specification. Record
%all personnel present during the execution including testers, operators, and observers. Also indicate the function
%of each individual.

The tests described here have been conducted by Samuel Kortas (EDF). The
execution procedure is described in specification
\refexpdoc{wp31-api-ts01-zephyr01-tds01}.

%(2)
\paragraph{Procedure Results}
%For each execution, record the visually observable results (e.g., error messages generated, aborts, and
%requests for operator action). Also record the location of any output (e.g., reel number). Record the successful
%or unsuccessful execution of the test.


Installation and set up of the boost and saga libraries did 
not reveal any other contingencies than the ones presented
in \refexpdoc{api:contingencies}. Summary: The installation has been completed successfully.


\subsubsection{Execution of the test program}

Calling the test program several times, it produced the following output :

\begin{lstlisting}
 NOT RUN! Job state is: New      
 JobID: [any://localhost]-[rendvous-fn1:9205]
 Just RUN! Job state is: Running  
 SUSPENDED! Job state is: Suspended
 RESUMED! Job state is: Running  
 NOT DONE! Job state is: Running  
 AWAITED! Job state is: Running  
 should be DONE by now! Job state is: Running  
 JobID: [any://localhost]-[rendvous-fn1:9385]
 Just RUN! Job state is: Running  
 CANCELED! Job state is: Canceled 
\end{lstlisting}

or this one alternatively

\begin{lstlisting}
 NOT RUN! Job state is: New      
 JobID: [any://localhost]-[rendvous-fn1:26007]
 Just RUN! Job state is: Running  
 SUSPENDED! Job state is: Suspended
 RESUMED! Job state is: Running  
 NOT DONE! Job state is: Running  
 terminate called after throwing an instance of 'boost::process::system_error'
  what():  boost::process::child::wait: waitpid(2) failed: No child processes
 Abandon
\end{lstlisting}


We checked that the job submitted was actually running on the 
machine by browsing the results produced by Zephyr stored
into a ``fort.66'' produced in the same directory as the
test program.


%(3)
%\paragraph{Environmental Information}
%Record any environmental conditions specific to this entry (e.g., hardware substitutions).

%(4)
\subsubsection{Anomalous Events}
%Record what happened before and after an unexpected event occurred (e.g., A summary display was
%requested and the correct screen displayed, but response seemed unusually long. A repetition produced the
%same prolonged response). Record circumstances surrounding the inability to begin execution of a test
%procedure or failure to complete a test procedure (e.g., a power failure or system software problem).

\begin{itemize}
 \item The ``exception error'' encountered in the alternative result
       obtained could be explained by the termination of the executed job
       before probing its status. The job disappeared, its
       status is therefore not readable any more and throwing the
       exception. \\
       {\bf Still this behavior does not agree with the Saga specification}
       stating that the status of the terminated job should have been
       ``Done''.
 \item In the ``regular'' output, the job status also remains
       ``running'' after a call to the wait() method. In fact
       in agreement with the Saga specification, we understand that
       the status should either be ``canceled'', ``done'' or ``failed''.
\end{itemize}





%(5)
%\subsubsection{Incident Report Identifiers}
%Record the identifier of each test incident report, whenever one is generated.
 Summary: Questions on these two points have been forwarded to the WP
3.1 team. But, for the moment we consider {\bf the test as
failed}. Still, though the tests could not be run through the XtreemOS
execution environment (SAGA API has not been ported on XtreemOS yet),
one could easily prepare, start, monitor, end or cancel a job, which
is very encouraging.  

