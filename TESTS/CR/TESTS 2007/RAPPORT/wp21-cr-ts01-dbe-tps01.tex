%%%%%%%%%%%%%%%%%%%%%%%%%%%%%%%%%%%%%%%%%
% Template for describing the test procedure specification.
% structure and annotations are quoted from IEEE Std 829-1998
%%%%%%%%%%%%%%%%%%%%%%%%%%%%%%%%%%%%%%%%%



\starttps{wp21-cr-ts01-dbe-tps01}\\
%Specify the unique identifier assigned to this test procedure specification.
%\noindent\textbf{Test procedure specification identifier: wp21-cr-ts01-dbe-tps01}\\
\noindent\textbf{Test log reference: \refexpdoc{wp21-cr-ts01-dbe-tl01}} \\%insert file name here without extension '.tex'
\noindent\textbf{Test design specification reference: \refexpdoc{wp21-cr-ts01-dbe-tds01}} %Supply a reference (identifier) to the associated test design specification.

%Purpose:
%To specify the steps for executing a set of test cases or, more generally, the steps used to analyze a software
%item in order to evaluate a set of features.

%Outline:
%A test procedure specification shall have the following structure:
%a) Test procedure specification identifier.
%b) Purpose;
%c) Special requirements;
%d) Procedure steps.
%The sections shall be ordered in the specified sequence. Additional sections, if required, may be included at
%the end. If some or all of the content of a section is in another document, then a reference to that material
%may be listed in place of the corresponding content. The referenced material must be attached to the test
%procedure specification or available to users of the procedure specification.
%Details on the content of each section are contained in the following subclauses.


\subsubsection{Purpose}

The purpose of the test is to see if the checkpointing and restart is able to handle simple Java applications

\subsubsection{Procedure Steps}
%Include the following steps as applicable.

%\paragraph{Log}
%Describe any special methods or formats for logging the results of test execution, the incidents observed, and
%any other events pertinent to the test (see Clauses 9 and 10).

\paragraph{Set Up}
%Describe the sequence of actions necessary to prepare for execution of the procedure.
\begin{itemize}
\item Download the latest BLCR module v0.6.1 from:\\
	http://ftg.lbl.gov/CheckpointRestart/CheckpointDownloads.shtml
\item Configure and make the module.
\item Install the module
\end{itemize}

\paragraph{Start}

\begin{itemize}
\item Check that the BLCR modules are inserted properly. {\tt lsmod } will display blcr, blcr\_vmadump and blcr\_imports if they are there.
\item Run {\tt make insmod} command from the installation directory if they are not there already (requires root).
\item Start the Java application using {\tt cr\_run }. The entire command will look something like this:\\
	{\tt cr\_run java -jar fileTest.jar file1.txt file2.txt}
\end{itemize}


\paragraph{Proceed}

\begin{itemize}
\item While the application is running determine its pid using {\tt ps }.
\item Run the BLCR checkpoint command\\
	{\tt cr\_checkpoint [options] PID} 
\item Kill the application using SIGKILL instead of SIGTERM, that should say to the virtual machine that the application exited correctly.
\item Restart the application from the checkpoint using the command\\
	{\tt cr\_restart context.PID }.
\item Observe that the application finishes correctly.
\end{itemize}

\paragraph{Measure}
The test will succeed if after killing the application, it can restart from the checkpointed status. The applications is writing an 
auto-incremented number each second. If we checkpoint at second 4, and kill the application at second 10, we expect that the application
restart showing 5 in the screen an not starting from the beginning of continuing from 10.

\paragraph{Shut Down}
We will kill the application with the -9 parameter

\paragraph{Restart}
We will execute the restart procedure

\paragraph{Stop}
%Describe the actions necessary to bring execution to an orderly halt.
The Java application never stops

%\paragraph{Wrap Up}
%Describe the actions necessary to restore the environment.

\paragraph{Contingencies}
%Describe the actions necessary to deal with anomalous events that may occur during execution.
There where not contingencies.
