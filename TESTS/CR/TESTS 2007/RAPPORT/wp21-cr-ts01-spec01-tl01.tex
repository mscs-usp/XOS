%%%%%%%%%%%%%%%%%%%%%%%%%%%%%%%%%%%%%%%%%
% Template for describing the test log.
% structure and annotations are quoted from IEEE Std 829-1998
%%%%%%%%%%%%%%%%%%%%%%%%%%%%%%%%%%%%%%%%%

\starttl{wp21-cr-ts01-spec01-tl01}
%Specify the unique identifier assigned to this test log.

%Purpose:
%To provide a chronological record of relevant details about the execution of tests.

%Outline:
%A test log shall have the following structure:
%a) Test log identifier;
%b) Description;
%c) Activity and event entries.
%The sections shall be ordered in the specified sequence. Additional sections may be included at the end. If
%some or all of the content of a section is in another document, then a reference to that material may be listed
%in place of the corresponding content. The referenced material must be attached to the test log or available to
%users of the log.
%Details on the content of each section are contained in the following subclauses.


\subsubsection{Description}
%Information that applies to all entries in the log except as specifically noted in a log entry should be included
%here. The following information should be considered:
%a) Identify the items being tested including their version/revision levels. For each of these items, supply
%a reference to its transmittal report, if it exists.
%b) Identify the attributes of the environments in which the testing is conducted. Include facility identifi-
%cation, hardware being used (e.g., amount of memory being used, CPU model number, and number
%and model of tape drives, and/or mass storage devices), system software used, and resources available
%(e.g., the amount of memory available).
Test case \refexpdoc{wp21-cr-ts01-spec01-tcs01} was executed according to procedure \refexpdoc{wp21-cr-ts01-spec01-tps01}.

\subsubsection{Activity and Event Entries}
%For each event, including the beginning and end of activities, record the occurrence date and time along with
%the identity of the author.
%The information in (1) through (5) should be considered:
This test was run on the 26th November 2007 by BSC.

%(1) 
\paragraph{Execution Description}
%Record the identifier of the test procedure being executed and supply a reference to its specification. Record
%all personnel present during the execution including testers, operators, and observers. Also indicate the function
%of each individual.
\refexpdoc{wp21-cr-ts01-spec01-tps01}

%(2)
\paragraph{Procedure Results}
~\\
%For each execution, record the visually observable results (e.g., error messages generated, aborts, and
%requests for operator action). Also record the location of any output (e.g., reel number). Record the successful
%or unsuccessful execution of the test.

\begin{lstlisting}
Results using term signal:
# cr_checkpoint --term <PID>
# cr_restart context.<PID>
Restart failed: No such file or directory
comment: the file context.<PID> was there.

First attempt using kill signal:
# cr_checkpoint --kill <PID>
# cr_restart context.<PID>
Restart failed: Permission denied
comment: the file context.<PID> had the correct owner and access permissions for the user.
\end{lstlisting}

After editing the /etc/ncsd.conf file and disabling name caching for passwd/groups the checkpoint command with the kill signal was reran and it completed successfully.


%(3)
%\paragraph{Environmental Information}
%Record any environmental conditions specific to this entry (e.g., hardware substitutions).


%(4)
\subsubsection{Anomalous Events}
%Record what happened before and after an unexpected event occurred (e.g., A summary display was
%requested and the correct screen displayed, but response seemed unusually long. A repetition produced the
%same prolonged response). Record circumstances surrounding the inability to begin execution of a test
%procedure or failure to complete a test procedure (e.g., a power failure or system software problem).
There were some failures to restart the checkpoints.

%(5)
\subsubsection{Incident Report Identifiers}
%Record the identifier of each test incident report, whenever one is generated.
The incident report for these tests is found in \refexpdoc{wp21-cr-ts01-spec01-tir01}.