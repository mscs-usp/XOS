%%%%%%%%%%%%%%%%%%%%%%%%%%%%%%%%%%%%%%%%%
% Template for describing the test log.
% structure and annotations are quoted from IEEE Std 829-1998
%%%%%%%%%%%%%%%%%%%%%%%%%%%%%%%%%%%%%%%%%

\starttl{wp21-cr-ts01-dbe-tl02}
%Specify the unique identifier assigned to this test log.


%Purpose:
%To provide a chronological record of relevant details about the execution of tests.

%Outline:
%A test log shall have the following structure:
%a) Test log identifier;
%b) Description;
%c) Activity and event entries.
%The sections shall be ordered in the specified sequence. Additional sections may be included at the end. If
%some or all of the content of a section is in another document, then a reference to that material may be listed
%in place of the corresponding content. The referenced material must be attached to the test log or available to
%users of the log.
%Details on the content of each section are contained in the following subclauses.


\subsubsection{Description}
Tests where performed by T6. No more hardware than a linux computer was needed.

\subsubsection{Activity and Event Entries}
All tests were performed on the 8 November 2007 by T6.


%(1) 
\paragraph{Execution Description}
%Record the identifier of the test procedure being executed and supply a reference to its specification. Record
%all personnel present during the execution including testers, operators, and observers. Also indicate the function
%of each individual.
see \refexpdoc{wp21-cr-ts01-dbe-tps02}


%(2)
\paragraph{Procedure Results}
The test failed. Even when the application was able to restart, communication sockets where not valid anymore, and the communication with the clients was lost.

%(3)
%\paragraph{Environmental Information}

%(4)
\subsubsection{Anomalous Events}
The server socket cannot be opened when the application is restarted.


%(5)
\subsubsection{Incident Report Identifiers}
%Record the identifier of each test incident report, whenever one is generated.
\refexpdoc{wp21-cr-ts01-dbe-tir02}
