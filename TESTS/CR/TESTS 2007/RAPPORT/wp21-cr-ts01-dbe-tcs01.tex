%%%%%%%%%%%%%%%%%%%%%%%%%%%%%%%%%%%%%%%%%
% Template for describing the test case specification.
% structure and annotations are quoted from IEEE Std 829-1998
%%%%%%%%%%%%%%%%%%%%%%%%%%%%%%%%%%%%%%%%%


\starttcs{wp21-cr-ts01-dbe-tcs01}{Checkpointing a simple Java file-writing application}\\
%\noindent\textbf{Test case specification identifier: wp21-cr-ts01-dbe-tcs01}%insert file name here without extension '.tex'
%Purpose:
%To define a test case identified by a test design specification.
The main purpose of this test is to know which operations can be checkpointed and restarted and which ones can not. For instance, the
dbe application is a very intensive application in terms of files, threads and sockets. It also uses advanced Java features such as
dynamic class loading. We will test progressively if these main features can be checkpointed and restarted easily or if them present
problems while restarting.

%Outline:
%A test case specification shall have the following structure:
%a) Test case specification identifier;
%b) Test items;
%c) Input specifications;
%d) Output specifications;
%e) Environmental needs;
%f) Special procedural requirements;
%g) Intercase dependencies.
%The sections shall be ordered in the specified sequence. Additional sections may be included at the end. If
%some or all of the content of a section is in another document, then a reference to that material may be listed
%in place of the corresponding content. The referenced material must be attached to the test case specification
%or available to users of the case specification.
%Since a test case may be referenced by several test design specifications used by different groups over a long
%time period, enough specific information must be included in the test case specification to permit reuse.
%Details on the content of each section are contained in the following subclauses.


\subsubsection{Test Items}
In an infinite loop this application opens two files stored in the same directory where the application is and writes a string on them. The file names are passed as an argument to the program.

The string to be written in each file is a number that increments its value in each iteration. This number is stored in a static variable in the class. Its initial value is 1.

\subsubsection{Input Specifications}

The input of the application are the names of the files we want to write to.

\subsubsection{Output Specifications}

Two files with a list of numbers form 1 to X. One number is written each second.

\subsubsection{Environmental Needs}

\paragraph{Hardware}

A single x86 machine is required.

\paragraph{Software}

The following software was used:
\begin{itemize}
\item Linux kernel 2.6.14-generic
\item JRE 1.5.0\_10 from Sun
\item BLCR 0.6.1
\end{itemize}

