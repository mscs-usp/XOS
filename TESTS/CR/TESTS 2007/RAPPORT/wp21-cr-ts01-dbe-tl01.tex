%%%%%%%%%%%%%%%%%%%%%%%%%%%%%%%%%%%%%%%%%
% Template for describing the test log.
% structure and annotations are quoted from IEEE Std 829-1998
%%%%%%%%%%%%%%%%%%%%%%%%%%%%%%%%%%%%%%%%%

\starttl{wp21-cr-ts01-dbe-tl01}
%Specify the unique identifier assigned to this test log.
%\noindent\textbf{Test log identifier: wp21-cr-ts01-dbe-tl01}%insert file name here without extension '.tex'

%Purpose:
%To provide a chronological record of relevant details about the execution of tests.

%Outline:
%A test log shall have the following structure:
%a) Test log identifier;
%b) Description;
%c) Activity and event entries.
%The sections shall be ordered in the specified sequence. Additional sections may be included at the end. If
%some or all of the content of a section is in another document, then a reference to that material may be listed
%in place of the corresponding content. The referenced material must be attached to the test log or available to
%users of the log.
%Details on the content of each section are contained in the following subclauses.


\subsubsection{Description}
Tests where preformed by T6. No more hardware than a computer was needed

\subsubsection{Activity and Event Entries}
All tests were performed on the 8 November 2007 by T6.

%(1) 
\paragraph{Execution Description}
see \refexpdoc{wp21-cr-ts01-dbe-tps01}

%(2)
\paragraph{Procedure Results}
The test was executed successfully. The program was re\-started after a KILL signal, following the execution from the status saved in 
the checkpoint file.

This is the output of the application while checkpointing:
writing 1
writing 2
writing 3
writing 4 (we checkpoint here)
writing 5
writing 6 (we kill here)
killed

After restarting, we can see the output
writing 5
writing 6
writing 7
writing 8
...

Output files involved in the test contain the correct list of numbers form 1 to n.


%(3)
\paragraph{Environmental Information}
%Record any environmental conditions specific to this entry (e.g., hardware substitutions).
The test was executed in a laptop computer running Linux (kernel version 2.6.22-14) 

%(4)
\subsubsection{Anomalous Events}
There were not anomalous events. The program restarted and the executed continued from the checkpointing.

%(5)
\subsubsection{Incident Report Identifiers}
%Record the identifier of each test incident report, whenever one is generated.
There were not incidents, the application could restart as expected from tho checkpointed status.
