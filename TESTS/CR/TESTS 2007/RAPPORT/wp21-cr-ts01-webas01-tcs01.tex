%%%%%%%%%%%%%%%%%%%%%%%%%%%%%%%%%%%%%%%%%
% Template for describing the test case specification.
% structure and annotations are quoted from IEEE Std 829-1998
%%%%%%%%%%%%%%%%%%%%%%%%%%%%%%%%%%%%%%%%%
\starttcs{wp21-cr-ts01-webas01-tcs01}{CR with Simple Applications}

%\noindent\textbf{Test case specification identifier: }%insert file name here without extension '.tex'

%Purpose:
%To define a test case identified by a test design specification.

%Outline:
%A test case specification shall have the following structure:
%a) Test case specification identifier;
%b) Test items;
%c) Input specifications;
%d) Output specifications;
%e) Environmental needs;
%f) Special procedural requirements;
%g) Intercase dependencies.
%The sections shall be ordered in the specified sequence. Additional sections may be included at the end. If
%some or all of the content of a section is in another document, then a reference to that material may be listed
%in place of the corresponding content. The referenced material must be attached to the test case specification
%or available to users of the case specification.
%Since a test case may be referenced by several test design specifications used by different groups over a long
%time period, enough specific information must be included in the test case specification to permit reuse.
%Details on the content of each section are contained in the following subclauses.


\subsubsection{Test Items}
%Identify and brie��y describe the items and features to be exercised by this test case.
%For each item, consider supplying references to the following test item documentation:
%a) Requirements specification;
%b) Design specification;
%c) Users guide;
%d) Operations guide;
%e) Installation guide.
We will test the checkpointing and restart module of WP2.1. To this end, we will first use two simple examples. The first counting example is provided by BLCR. The second example is the simple Linux editor \emph{gedit}.

\subsubsection{Input Specifications}
%Specify each input required to execute the test case. Some of the inputs will be specified by value (with
%tolerances where appropriate), while others, such as constant tables or transaction files, will be specified by
%name. Identify all appropriate databases, files, terminal messages, memory resident areas, and values passed
%by the operating system.
%Specify all required relationships between inputs (e.g., timing).
The input will be on the one hand the simple counting example from BLCR. The second input will be the gedit application.

The counting program does not require any input as this application simply counts from 1 to 100. The input for the \emph{gedit} application is ''qwert''.

\subsubsection{Output Specifications}
%Specify all of the outputs and features (e.g., response time) required of the test items. Provide the exact value
%(with tolerances where appropriate) for each required output or feature.
The output for the counting program is that the remaining numbers up to 100 are correctly shown.

After restarting the editor \emph{gedit}, the content ''qwert'' should be displayed. Furthermore, it must be possible to modify the text and save it properly.

\subsubsection{Environmental Needs}

\paragraph{Hardware}
%Specify the characteristics and configurations of the hardware required to execute this test case (e.g.,
%132 character 24 line CRT).
A single x86 machine is required.

\paragraph{Software}
%Specify the system and application software required to execute this test case. This may include system software
%such as operating systems, compilers, simulators, and test tools. In addition, the test item may interact
%with application software.
The following software is required:
\begin{itemize}
\item Linux kernel 2.6.16 or higher
\item BLCR 0.6.1 with the XtreemOS modifications
\item simple counting program provided by BLCR and a simple text editor \emph{gedit}.
\end{itemize}
