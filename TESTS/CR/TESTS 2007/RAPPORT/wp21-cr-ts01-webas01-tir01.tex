%%%%%%%%%%%%%%%%%%%%%%%%%%%%%%%%%%%%%%%%%
% Template for describing the test incident report.
% structure and annotations are quoted from IEEE Std 829-1998
%%%%%%%%%%%%%%%%%%%%%%%%%%%%%%%%%%%%%%%%%
\starttir{wp21-cr-ts01-webas01-tir01}{Text editor does not restart}


%Specify the unique identifier assigned to this test incident report.
%\noindent\textbf{Test incident report identifier: %insert file name here without extension '.tex'


%Purpose:
%To document any event that occurs during the testing process that requires investigation.

%Outline:
%A test incident report shall have the following structure:
%a) Test incident report identifier;
%b) Summary;
%c) Incident description;
%d) Impact.
%The sections shall be ordered in the specified sequence. Additional sections may be included at the end. If
%some or all of the content of a section is in another document, then a reference to that material may be listed
%in place of the corresponding content. The referenced material must be attached to the test incident report or
%available to users of the incident report.
%Details on the content of each section are contained in the following subclauses.



\subsubsection{Summary}
%Summarize the incident. Identify the test items involved indicating their version/revision level. References to
%the appropriate test procedure specification, test case specification, and test log should be supplied.
The simple text editor \emph{gedit} could not be restarted from the checkpoint. 

\subsubsection{Incident Description}
%Provide a description of the incident. This description should include the following items:
%a) Inputs;
%b) Expected results;
%c) Actual results;
%d) Anomalies;
%e) Date and time;
%f) Procedure step;
%g) Environment;
%h) Attempts to repeat;
%i) Testers;
%j) Observers.
%Related activities and observations that may help to isolate and correct the cause of the incident should be
%included (e.g., describe any test case executions that might have a bearing on this particular incident and any
%variations from the published test procedure).

After a checkpointing of the text editor \emph{gedit}, the editor could not be restarted correctly. The output on the command line gives the impression that the checkpointing itself was successful. However, the failed restart might result from a checkpoint that was not generated correctly. Thus, it is the task of the development groups to find the real reason.

\subsubsection{Impact}
%If known, indicate what impact this incident will have on test plans, test design specifications, test procedure
%specifications, or test case specifications.
The text editor could not be restarted from the checkpoint. Thus, the application could not be used at all after the checkpointing.
