%%%%%%%%%%%%%%%%%%%%%%%%%%%%%%%%%%%%%%%%%
% Template for describing the test design specification.
% structure and annotations are quoted from IEEE Std 829-1998
%%%%%%%%%%%%%%%%%%%%%%%%%%%%%%%%%%%%%%%%%


\starttds{wp21-cr-ts01-spec01-tds01}\\
%Specify the unique identifier assigned to this test design specification. 
\noindent\textbf{Test plan reference: \refexpdoc{wp21-cr-ts01-tp}} %Supply a reference (identifier) to the associated test plan.

%Purpose
%To specify refinements of the test approach and to identify the features to be tested by this design and its
%associated tests.

%Outline:
%A test design specification shall have the following structure:
%a) Test design specification identifier;
%b) Features to be tested;
%c) Approach refinements;
%d) Test identification;
%e) Feature pass/fail criteria.
%The sections shall be ordered in the specified sequence. Additional sections may be included at the end. If
%some or all of the content of a section is in another document, then a reference to that material may be listed
%in place of the corresponding content. The referenced material must be attached to the test design specification
%or available to users of the design specification.
%Details on the content of each section are contained in the following subclauses.

\subsubsection{Features to be Tested}
%Identify the test items and describe the features and combinations of features that are the object of this
%design specification. Other features may be exercised, but need not be identified.
%For each feature or feature combination, a reference to its associated requirements in the item requirement
%specification or design description should be included.
The testing of the checkpointing and restarting of a Java application
is the objective of these tests. More specifically, the tests apply to
requirements \ref{req:restart_mimic}, \ref{req:check_initiation},
\ref{req:check_kernel} and \ref{req:check_save}.


\subsubsection{Approach Refinements}
%Specify refinements to the approach described in the test plan. Include specific test techniques to be used.
%The method of analyzing test results should be identified (e.g., comparator programs or visual inspection).
%Specify the results of any analysis that provides a rationale for test case selection. For example, one might
%specify conditions that permit a determination of error tolerance (e.g., those conditions that distinguish valid
%inputs from invalid inputs).
%Summarize the common attributes of any test cases. This may include input constraints that must be true for
%every input in the set of associated test cases, any shared environmental needs, any shared special procedural
%requirements, and any shared case dependencies.
Since these are the first tests that are being conducted on the checkpointer, the SPECweb application will not be used at this stage. In order to get early and useful feedback we will attempt to checkpoint and restart a simple multi-threaded Java application and based on the results determine the viability of running the tests on the SPECweb application. The results obtained will provide information that will apply to all Java applications.
Two separate cases will be considered:
\begin{itemize}
\item Testing with the current version of BLCR (0.6.1)
\item Testing with the updated version of BLCR for XtreemOS
\end{itemize}


\subsubsection{Test Identification}
The test case with the current version of BLCR is \refexpdoc{wp21-cr-ts01-spec01-tcs01} while the test case for the version for XtreemOS is \refexpdoc{wp21-cr-ts01-spec01-tcs02}.
%List the identifier and a brief description of each test case associated with this design. A particular test case
%may be identified in more than one test design specification. List the identifier and a brief description of each
%procedure associated with this test design specification.



\subsubsection{Feature Pass/Fail Criteria}
%Specify the criteria to be used to determine whether the feature or feature combination has passed or failed.
If the application can be checkpointed and restarted with no noticeable differences it will have passed. Anything else will be considered as a failure.
