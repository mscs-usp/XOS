%%%%%%%%%%%%%%%%%%%%%%%%%%%%%%%%%%%%%%%%%
% Template for describing the test procedure specification.
% structure and annotations are quoted from IEEE Std 829-1998
%%%%%%%%%%%%%%%%%%%%%%%%%%%%%%%%%%%%%%%%%
\starttps{wp21-cr-ts01-webas01-tps01}



%Specify the unique identifier assigned to this test procedure specification.
%\noindent\textbf{Test log identifier: } %insert file name here without extension '.tex'
\noindent\textbf{Test design specification reference: \refexpdoc{wp21-cr-ts01-webas01-tds01}}\\
\noindent\textbf{Test case specification reference: \refexpdoc{wp21-cr-ts01-webas01-tcs01}}\\
\noindent\textbf{Test log identifier: \refexpdoc{wp21-cr-ts01-webas01-tl01}}
 %Supply a reference (identifier) to the associated test design specification.

%Purpose:
%To specify the steps for executing a set of test cases or, more generally, the steps used to analyze a software
%item in order to evaluate a set of features.

%Outline:
%A test procedure specification shall have the following structure:
%a) Test procedure specification identifier.
%b) Purpose;
%c) Special requirements;
%d) Procedure steps.
%The sections shall be ordered in the specified sequence. Additional sections, if required, may be included at
%the end. If some or all of the content of a section is in another document, then a reference to that material
%may be listed in place of the corresponding content. The referenced material must be attached to the test
%procedure specification or available to users of the procedure specification.
%Details on the content of each section are contained in the following subclauses.


\subsubsection{Purpose}
%Describe the purpose of this procedure. If this procedure executes any test cases, provide a reference for each
%of them.
%In addition, provide references to relevant sections of the test item documentation (e.g., references to usage
%procedures).
The purpose of these tests is to see how the checkpointing and restart module will handle a simple counting program as well as a simple text editor \emph{gedit}.

\subsubsection{Procedure Steps}
%Include the following steps as applicable.

\paragraph{Installation}
The test requires a proper Linux system without further modifications. The setup of the checkpointing and restarting functionalities is described under setup.

\paragraph{Log}
%Describe any special methods or formats for logging the results of test execution, the incidents observed, and
%any other events pertinent to the test (see Clauses 9 and 10).
The output of the counting program is logged into a file using the command line. The output of the editor \emph{gedit} is simply described.

\paragraph{Set Up}
%Describe the sequence of actions necessary to prepare for execution of the procedure.
The setup is as follows.\\
\begin{lstlisting}
Install BLCR:
 	svn+ssh://<login>@scm.gforge.inria.fr/~
 	svn/xtreemos/WP2.1/T2/1.4/

 # cd blcr-<VERSION>
 # chmod 777 autogen.sh
 # ./autogen.sh
 # mkdir builddir
 # cd builddir
 #../configure
 # make
 # make install

Load BLCR modules

# /sbin/insmod 
/usr/local/lib/blcr/2.6.12-1.234/blcr_imports.ko
# /sbin/insmod 
/usr/local/lib/blcr/2.6.12-1.234/blcr_vmadump.ko
# /sbin/insmod 
/usr/local/lib/blcr/2.6.12-1.234/blcr.ko

Check that modules have been loaded correctly

# lsmod | grep blcr

Should produce: (if not, try reloading modules again)

blcr                   67728  0
blcr_vmadump           37528  1 blcr
blcr_imports           25856  2 blcr,blcr_vmadump
\end{lstlisting}

\paragraph{Start}
For the counting example:\\
\begin{lstlisting}
(Open new shell - Shell A)
# cd blcr-0.5.3/builddir/counting
# cr_run  ./counting

Counting demo starting with <PID>
Count = 0
Count = 1
Count = 2
Count = 3

(Open new shell - Shell B)

# cr_checkpoint -term <PID>
\end{lstlisting}
For the simple text editor \emph{gedit}:\\
\begin{lstlisting}
#cr_run ./gedit
\end{lstlisting}


\paragraph{Proceed}

For the counting example:
\begin{lstlisting}
(In Shell A)
... 
Count = 3
Terminated

#cr_restart context.<PID>

Count = 4
Count = 5
Count = 6
Count = 7
\end{lstlisting}
For the text editor example, we have four different options for a checkpointing. In general, these options are also valid for the counting example but not necessary.
\begin{lstlisting}
Option 1:
#cr_checkpoint -kill <PID>
#cr_restart context.<PID>

Option 2:
#cr_checkpoint --save-all --kill <PID>
#cr_restart context.<PID>

Option 3:
#cr_checkpoint --save-exe --kill <PID>
#cr_restart context.<PID>

Option 4:
# cr_checkpoint --kill --save-private --save-shared <PID>
# cr_restart context.<PID>
\end{lstlisting}

\paragraph{Measure}
%Describe how the test measurements will be made (e.g., describe how remote terminal response time is to be
%measured using a network simulator).
The two tests do not require special measurement setups.

\paragraph{Shut Down}
%Describe the actions necessary to suspend testing, when unscheduled events dictate.
All tests can be shut down by using the process kill options provided by Linux.

\paragraph{Restart}
%Identify any procedural restart points and describe the actions necessary to restart the procedure at each of
%these points.
A restart is in both cases only meaningful from the beginning. Thus, restart in this case means performing the test again right from the beginning.

\paragraph{Stop}
%Describe the actions necessary to bring execution to an orderly halt.
The counting example stops after having counted to 100. The text editor has to be stopped manually from the corresponding menu.

\paragraph{Wrap Up}
%Describe the actions necessary to restore the environment.
Both tests do not require an explicit wrap up. However, it is meaningful to remove the stored checkpoints after the test in order to keep the hard disc clean.

\paragraph{Contingencies}
%Describe the actions necessary to deal with anomalous events that may occur during execution.
In case of anomalies, the applications should simply be restarted.
