%%%%%%%%%%%%%%%%%%%%%%%%%%%%%%%%%%%%%%%%%
% Template for describing the test procedure specification.
% structure and annotations are quoted from IEEE Std 829-1998
%%%%%%%%%%%%%%%%%%%%%%%%%%%%%%%%%%%%%%%%%



\starttps{wp21-cr-ts01-spec01-tps01}\\
%Specify the unique identifier assigned to this test procedure specification.
\noindent\textbf{Test design specification reference: \refexpdoc{wp21-cr-ts01-spec01-tds01}} \\%Supply a reference (identifier) to the associated test design specification.
\noindent\textbf{Test case specification reference: \refexpdoc{wp21-cr-ts01-spec01-tcs01}} \\
\noindent\textbf{Test log identifier: \refexpdoc{wp21-cr-ts01-spec01-tl01}} %insert file name here without extension '.tex'

%Purpose:
%To specify the steps for executing a set of test cases or, more generally, the steps used to analyze a software
%item in order to evaluate a set of features.

%Outline:
%A test procedure specification shall have the following structure:
%a) Test procedure specification identifier.
%b) Purpose;
%c) Special requirements;
%d) Procedure steps.
%The sections shall be ordered in the specified sequence. Additional sections, if required, may be included at
%the end. If some or all of the content of a section is in another document, then a reference to that material
%may be listed in place of the corresponding content. The referenced material must be attached to the test
%procedure specification or available to users of the procedure specification.
%Details on the content of each section are contained in the following subclauses.


\subsubsection{Purpose}
%Describe the purpose of this procedure. If this procedure executes any test cases, provide a reference for each
%of them.
%In addition, provide references to relevant sections of the test item documentation (e.g., references to usage
%procedures).
The purpose of this test is to see how the checkpointer module will handle a multi-threaded Java application.

\subsubsection{Procedure Steps}
%Include the following steps as applicable.
\paragraph{Set Up}
%Describe the sequence of actions necessary to prepare for execution of the procedure.
\begin{itemize}
\item Download the latest (as of 25-09-07) BCLR module v0.6.1 from\newline
	\href{http://ftg.lbl.gov/CheckpointRestart/CheckpointDownloads.shtml}{http://ftg.lbl.gov/CheckpointRestart/CheckpointDownloads.shtml}.
\item Untar the package and enter the installation directory.
\item Configure for the version of Linux currently being used with a command such as\\ {\tt ./configure --with-linux=/usr/src/linux-2.6.15}
\item Perform the {\tt make} command.
\item Check that the module has been created correctly; {\tt make insmod check }
\item If so, install the module; {\tt make install } (root password required)
\item Update the LD\_LIBRARY\_PATH if needed:\\
 \mbox{\footnotesize\tt export LD\_LIBRARY\_PATH=\$LD\_LIBRARY\_PATH:/usr/local/lib }
\end{itemize}

\paragraph{Start}
%Describe the actions necessary to begin execution of the procedure.
\begin{itemize}
\item Check that the BLCR modules are inserted properly. {\tt lsmod } will display blcr, blcr\_vmadump and blcr\_imports if they are there.
\item Run {\tt make insmod} command from the installation directory if they are not there already (requires root).
\item Start the Java application using {\tt cr\_run }. The entire command will look something like this:\\{\tt cr\_run java -jar myThreadedTest.jar}
\end{itemize}


\paragraph{Proceed}
%Describe any actions necessary during execution of the procedure.
\begin{itemize}
\item While the application is running determine its PID using {\tt ps }.
\item Run the BLCR checkpoint command\\
	{\tt cr\_checkpoint [options] PID}\\
	where OPTIONS should be a Linux signal such as term or kill, and the PID used is the one previously determined.
\item Check that the checkpoint file was created in the current directory. It will be called 'context.PID'.
\item Restart the application from the checkpoint using the command\\
	{\tt cr\_restart context.PID }.
\item Observe that the application finishes correctly.
\end{itemize}

\paragraph{Shut Down}
%Describe the actions necessary to suspend testing, when unscheduled events dictate.
The severity of the error messages will dictate whether the test needs to be suspended.

\paragraph{Stop}
%Describe the actions necessary to bring execution to an orderly halt.
The Java application was written to stop after a predetermined time.
