%%%%%%%%%%%%%%%%%%%%%%%%%%%%%%%%%%%%%%%%%
% Template for describing the test log.
% structure and annotations are quoted from IEEE Std 829-1998
%%%%%%%%%%%%%%%%%%%%%%%%%%%%%%%%%%%%%%%%%
\starttl{wp21-cr-ts01-webas01-tl01}

%Specify the unique identifier assigned to this test log.
%\noindent\textbf{Test log identifier:}%insert file name here without extension '.tex'

%Purpose:
%To provide a chronological record of relevant details about the execution of tests.

%Outline:
%A test log shall have the following structure:
%a) Test log identifier;
%b) Description;
%c) Activity and event entries.
%The sections shall be ordered in the specified sequence. Additional sections may be included at the end. If
%some or all of the content of a section is in another document, then a reference to that material may be listed
%in place of the corresponding content. The referenced material must be attached to the test log or available to
%users of the log.
%Details on the content of each section are contained in the following subclauses.


\subsubsection{Description}
%Information that applies to all entries in the log except as specifically noted in a log entry should be included
%here. The following information should be considered:
%a) Identify the items being tested including their version/revision levels. For each of these items, supply
%a reference to its transmittal report, if it exists.
%b) Identify the attributes of the environments in which the testing is conducted. Include facility identifi-
%cation, hardware being used (e.g., amount of memory being used, CPU model number, and number
%and model of tape drives, and/or mass storage devices), system software used, and resources available
%(e.g., the amount of memory available).
The tests were performed by SAP. Furthermore, the requirements on the setup were rather small such that only a few MB memory was required and no special hardware at all.


\subsubsection{Activity and Event Entries}
%For each event, including the beginning and end of activities, record the occurrence date and time along with
%the identity of the author.
%The information in (1) through (5) should be considered:
All tests were performed on the 10th and 28th November 2007 by SAP.

%(1) 
\paragraph{Execution Description}
%Record the identifier of the test procedure being executed and supply a reference to its specification. Record
%all personnel present during the execution including testers, operators, and observers. Also indicate the function
%of each individual.
All tests were performed according to the test setup. Thus, a standard SuSE Linux Enterprise 10.3 was used. Thereafter, the checkpointing and restarting mechanism provided by WP2.1 was installed and finally, the tests executed.

%(2)
\paragraph{Procedure Results}
%For each execution, record the visually observable results (e.g., error messages generated, aborts, and
%requests for operator action). Also record the location of any output (e.g., reel number). Record the successful
%or unsuccessful execution of the test.
For the counting problem we have received a valid and correct output:
\begin{lstlisting}
(Open new shell - Shell A)
# cd blcr-0.5.3/builddir/counting
# cr_run  ./counting

Counting demo starting with pid 11539
Count = 0
Count = 1
Count = 2
Count = 3

(Open new shell - Shell B)

# cr_checkpoint -term 11539 

(In Shell A)

... 
Count = 3
Terminated

#cr_restart context.11539

Count = 4
Count = 5
Count = 6
Count = 7
\end{lstlisting}

For the text editor test, we have received an unusual result for all four options:
\begin{lstlisting}
Option 1:
# cr_checkpoint --kill <PID>
# cr_restart context.<PID>
Restart failed: No such file or directory
comment: the file context.<PID> was there.

Option 2:
# cr_checkpoint --save-all --kill <PID>
# cr_restart context.<PID>
Restart failed: No such file or directory
comment: the file context.<PID> was there.

Option 3:
#  cr_checkpoint --save-exe --kill <PID>
# cr_restart context.<PID>
Restart failed: No such file or directory
comment: the file context.<PID> was there.

Option 4:
# cr_checkpoint --kill --save-private --save-shared <PID>
# cr_restart context.<PID>
Restart failed: No such file or directory
comment: the file context.<PID> was there.
\end{lstlisting}

Therefore, we conclude that the checkpointing and restarting mechanism does not work properly for the text editor \emph{gedit}.

%(3)
\paragraph{Environmental Information}
%Record any environmental conditions specific to this entry (e.g., hardware substitutions).
We cannot associate special environmental information. The system setup was stable and did not change during our experiments.

%(4)
\subsubsection{Anomalous Events}
%Record what happened before and after an unexpected event occurred (e.g., A summary display was
%requested and the correct screen displayed, but response seemed unusually long. A repetition produced the
%same prolonged response). Record circumstances surrounding the inability to begin execution of a test
%procedure or failure to complete a test procedure (e.g., a power failure or system software problem).
The text editor example failed to restart. However, we cannot conclude for the
reason. This might be hidden in the generation of the checkpoint or in
the restart mechanism.

%(5)
\subsubsection{Incident Report Identifiers}
%Record the identifier of each test incident report, whenever one is generated.
The incident report can be found under the identifier: \refexpdoc{wp21-cr-ts01-webas01-tir01}.
