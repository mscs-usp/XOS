%%%%%%%%%%%%%%%%%%%%%%%%%%%%%%%%%%%%%%%%%
% Template for describing the test design specification.
% structure and annotations are quoted from IEEE Std 829-1998
%%%%%%%%%%%%%%%%%%%%%%%%%%%%%%%%%%%%%%%%%


\starttds{wp21-cr-ts01-dbe-tds01}\\
%Specify the unique identifier assigned to this test design specification. 
%\noindent\textbf{Test design specification identifier: wp21-cr-ts01-dbe-tds01}\\%insert file name here without extension '.tex'
\noindent\textbf{Test plan reference: \refexpdoc{wp21-cr-ts01-tp}} %Supply a reference (identifier) to the associated test plan.

%Purpose
%To specify refinements of the test approach and to identify the features to be tested by this design and its
%associated tests.

%Outline:
%A test design specification shall have the following structure:
%a) Test design specification identifier;
%b) Features to be tested;
%c) Approach refinements;
%d) Test identification;
%e) Feature pass/fail criteria.
%The sections shall be ordered in the specified sequence. Additional sections may be included at the end. If
%some or all of the content of a section is in another document, then a reference to that material may be listed
%in place of the corresponding content. The referenced material must be attached to the test design specification
%or available to users of the design specification.
%Details on the content of each section are contained in the following subclauses.

\subsubsection{Features to be Tested}
In order to test the checkpointing and restart we conduct two tests:
\begin{enumerate}
	\item checkpointing a basic Java application with a single thread and no socket connections. 
After killing the application we will try to restart it again in another machine and context.
	\item checkpointing a multi-thread application with several threads running at the same time
and several connections. After killing the application, we will try to restart it again on another
machine and environment (memory has been freed, connections have been closed, etc). 
\end{enumerate}

\subsubsection{Approach Refinements}
The initial tests in \refexpdoc{wp21-cr-ts01-dbe-tcs01} are not focused on the DBE 
application, that will be used in the second test, but on a more simple 
Java application.

This first test consists of a standalone application that does not open 
Input/Output connections, but stores local variables that change 
time.


\subsubsection{Test Identification}
The test case \refexpdoc{wp21-cr-ts01-dbe-tcs01} will perform the checkpointing and
restarting tests using a simple auto-increment program. The test case 
\refexpdoc{wp21-cr-ts01-dbe-tcs02} uses the DBE application


\subsubsection{Feature Pass/Fail Criteria}
The tests will pass if, the application ca be successfully checkpointed and restarted.

In detail, the auto-increment application passes if it maintains the last
number and can follow the normal process of auto-increment numbers. The second
test passes if after the restart, the DBE continues normally even if 
it lost ongoing transactions.


