%%%%%%%%%%%%%%%%%%%%%%%%%%%%%%%%%%%%%%%%%
% Template for describing the test case specification.
% structure and annotations are quoted from IEEE Std 829-1998
%%%%%%%%%%%%%%%%%%%%%%%%%%%%%%%%%%%%%%%%%


\starttcs{wp21-cr-ts01-dbe-tcs02}{Checkpointing a multithreaded Java application which uses sockets}
%\noindent\textbf{Test case specification identifier: wp21-cr-ts01-dbe-tcs02}%insert file name here without extension '.tex'

%Purpose:
%To define a test case identified by a test design specification.

%Outline:
%A test case specification shall have the following structure:
%a) Test case specification identifier;
%b) Test items;
%c) Input specifications;
%d) Output specifications;
%e) Environmental needs;
%f) Special procedural requirements;
%g) Intercase dependencies.
%The sections shall be ordered in the specified sequence. Additional sections may be included at the end. If
%some or all of the content of a section is in another document, then a reference to that material may be listed
%in place of the corresponding content. The referenced material must be attached to the test case specification
%or available to users of the case specification.
%Since a test case may be referenced by several test design specifications used by different groups over a long
%time period, enough specific information must be included in the test case specification to permit reuse.
%Details on the content of each section are contained in the following subclauses.


\subsubsection{Test Items}

We will test the DBE application. It is a multithreaded application that uses both, client and server sockets. Server sockets remain open and 
listening while the application is running. This could represent a problem when restarting, cause the kernel is controlling the open resources.

\subsubsection{Input Specifications}

The usual DBE start command, that includes the properties files where third party libraries are located

\subsubsection{Output Specifications}

The output is an infinite loop running application that waits for remote connections and processes them.

\subsubsection{Environmental Needs}

\paragraph{Hardware}

A single x86 machine is required.

\paragraph{Software}
The following software was used:
\begin{itemize}
\item Linux kernel 2.6.14-generic
\item JRE 1.5.0\_10 from Sun
\item BLCR 0.6.1 
\end{itemize}
