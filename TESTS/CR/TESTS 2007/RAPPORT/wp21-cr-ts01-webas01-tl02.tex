%%%%%%%%%%%%%%%%%%%%%%%%%%%%%%%%%%%%%%%%%
% Template for describing the test log.
% structure and annotations are quoted from IEEE Std 829-1998
%%%%%%%%%%%%%%%%%%%%%%%%%%%%%%%%%%%%%%%%%
\starttl{wp21-cr-ts01-webas01-tl02}

%Specify the unique identifier assigned to this test log.
%\noindent\textbf{Test log identifier: wp21-cr-ts01-webas01-tl02}%insert file name here without extension '.tex'

%Purpose:
%To provide a chronological record of relevant details about the execution of tests.

%Outline:
%A test log shall have the following structure:
%a) Test log identifier;
%b) Description;
%c) Activity and event entries.
%The sections shall be ordered in the specified sequence. Additional sections may be included at the end. If
%some or all of the content of a section is in another document, then a reference to that material may be listed
%in place of the corresponding content. The referenced material must be attached to the test log or available tou
%users of the log.
%Details on the content of each section are contained in the following subclauses.


\subsubsection{Description}
%Information that applies to all entries in the log except as specifically noted in a log entry should be included
%here. The following information should be considered:
%a) Identify the items being tested including their version/revision levels. For each of these items, supply
%a reference to its transmittal report, if it exists.
%b) Identify the attributes of the environments in which the testing is conducted. Include facility identifi-
%cation, hardware being used (e.g., amount of memory being used, CPU model number, and number
%and model of tape drives, and/or mass storage devices), system software used, and resources available
%(e.g., the amount of memory available).
The tests were performed by SAP using the SAP WebAS. The tests required a normal x86 machine with at least 4~GB of memory and 150~GB of hard disc space.

\subsubsection{Activity and Event Entries}
%For each event, including the beginning and end of activities, record the occurrence date and time along with
%the identity of the author.
%The information in (1) through (5) should be considered:
All tests were performed on the 23rd November 2007 by SAP.


%(1) 
\paragraph{Execution Description}
%Record the identifier of the test procedure being executed and supply a reference to its specification. Record
%all personnel present during the execution including testers, operators, and observers. Also indicate the function
%of each individual.
The tests were performed according to the test setup. Thus, a standard SuSE Linux Enterprise 10.3 was used. Thereafter, the checkpointing and restarting mechanism provided by WP2.1 was installed. Furthermore, the SAP WebAS was installed on the corresponding single machine. Finally, the tests were executed on the prepared machine.

%(2)
\paragraph{Procedure Results}
%For each execution, record the visually observable results (e.g., error messages generated, aborts, and
%requests for operator action). Also record the location of any output (e.g., reel number). Record the successful
%or unsuccessful execution of the test.
For the SAP WebAS, we have received an invalid result:
\begin{lstlisting}
# su benadm
% cr_run startsap

Checking BEN Database
------------------------------
 ABAP Database is not available via R3trans

Checking BEN Database
------------------------------

Starting SAP-Collector Daemon
------------------------------
ERROR: ld.so: object 'libcr.so.0' from LD_PRELOAD
cannot be preloaded: ignored.
ERROR: ld.so: object 'libpthread.so.0' from 
LD_PRELOAD cannot be preloaded: ignored.
17:34:24 08.11.2007   LOG: Effective User Id is root
***************************************************
* This is Saposcol Version COLL 20.93 640 - 
AMD/Intel x86_64 with Linux, PL:159, Nov 23 2006
* Usage:  saposcol -l: Start OS Collector
*         saposcol -k: Stop  OS Collector
*         saposcol -d: OS Collector Dialog Mode
*         saposcol -s: OS Collector Status
* The OS Collector (PID 4650) is already running .....
***************************************************
 saposcol already running
 Running /usr/sap/BEN/SYS/exe/run/startdb
Trying to start database ...
Log file: /usr/sap/BEN/benadm/startdb.log
ERROR: ld.so: object 'libcr.so.0' from LD_PRELOAD
cannot be preloaded: ignored.
ERROR: ld.so: object 'libpthread.so.0' from LD_PRELOAD
cannot be preloaded: ignored.
ERROR: ld.so: object 'libcr.so.0' from LD_PRELOAD 
cannot be preloaded: ignored.
ERROR: ld.so: object 'libpthread.so.0' from LD_PRELOAD 
cannot be preloaded: ignored.
ERROR: ld.so: object 'libcr.so.0' from LD_PRELOAD 
cannot be preloaded: ignored.
ERROR: ld.so: object 'libpthread.so.0' from LD_PRELOAD 
cannot be preloaded: ignored.
BEN database started
/usr/sap/BEN/SYS/exe/run/startdb completed successfully

Checking BEN Database
------------------------------
 ABAP Database is running

Starting SAP Instance DVEBMGS19
------------------------------
 Startup-Log is written to /usr/sap/BEN/benadm/startsap
 DVEBMGS19.log
 Instance on host bfssrv05 started
\end{lstlisting}
Obviously, some dynamic libraries could not be loaded using the BLCR approach. Therefore, the SAP WebAS cannot operate in the usual way. Thus, we did not reach the stage where we can use the checkpointing functionality itself.

%(3)
\paragraph{Environmental Information}
%Record any environmental conditions specific to this entry (e.g., hardware substitutions).
We cannot associate special environmental information. The system setup was stable and did not change during our experiments.

%(4)
\subsubsection{Anomalous Events}
%Record what happened before and after an unexpected event occurred (e.g., A summary display was
%requested and the correct screen displayed, but response seemed unusually long. A repetition produced the
%same prolonged response). Record circumstances surrounding the inability to begin execution of a test
%procedure or failure to complete a test procedure (e.g., a power failure or system software problem).

The SAP WebAS application failed to start correctly. Thus, no further experiments were possible.

%(5)
\subsubsection{Incident Report Identifiers}
%Record the identifier of each test incident report, whenever one is generated.
The incident report can be found under the identifier: \refexpdoc{wp21-cr-ts01-webas01-tir02}.
