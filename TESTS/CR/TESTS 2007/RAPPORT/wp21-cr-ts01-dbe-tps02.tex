%%%%%%%%%%%%%%%%%%%%%%%%%%%%%%%%%%%%%%%%%
% Template for describing the test procedure specification.
% structure and annotations are quoted from IEEE Std 829-1998
%%%%%%%%%%%%%%%%%%%%%%%%%%%%%%%%%%%%%%%%%



\starttps{wp21-cr-ts01-dbe-tps02}{Checkpointing multi-threaded Java application with sockets}\\
%Specify the unique identifier assigned to this test procedure specification.
%\noindent\textbf{Test procedure specification identifier: wp21-cr-ts01-dbe-tps02}\\ %insert file name here without extension '.tex'
\noindent\textbf{Test log reference: \refexpdoc{wp21-cr-ts01-dbe-tl02}}\\ %insert file name here without extension '.tex'
\noindent\textbf{Test design specification reference: \refexpdoc{wp21-cr-ts01-dbe-tds01}} %Supply a reference (identifier) to the associated test design specification.

%Purpose:
%To specify the steps for executing a set of test cases or, more generally, the steps used to analyze a software
%item in order to evaluate a set of features.

%Outline:
%A test procedure specification shall have the following structure:
%a) Test procedure specification identifier.
%b) Purpose;
%c) Special requirements;
%d) Procedure steps.
%The sections shall be ordered in the specified sequence. Additional sections, if required, may be included at
%the end. If some or all of the content of a section is in another document, then a reference to that material
%may be listed in place of the corresponding content. The referenced material must be attached to the test
%procedure specification or available to users of the procedure specification.
%Details on the content of each section are contained in the following subclauses.


\subsubsection{Purpose}

The purpose of this is to see how the checkpointer module will handle a multi-threaded Java application with server and client sockets open.

\subsubsection{Procedure Steps}
%Include the following steps as applicable.
We start the application in the standard way and, after few seconds we checkpoint it. We wait some more seconds and we kill it. After a sort period
of time, enough to allow the kernel to remove references to sockets that have been closed or destroyed, we try to restart the application
from the checkpointed status.

\paragraph{Log}
%Describe any special methods or formats for logging the results of test execution, the incidents observed, and
%any other events pertinent to the test (see Clauses 9 and 10).
The multithreaded application starts and most of the threads continue running from the checkpointing status. Some of them fail (controlled exceptions)
because they cannot read/write in their assigned sockets.

The application could restart fine if, after this error, the server socket is open again. Unfortunately, this kind of error in a server is
not expected, and the exception is simply reported, and the socket is not re\-opened of re\-binded again.

\paragraph{Set Up}
%Describe the sequence of actions necessary to prepare for execution of the procedure.
\begin{itemize}
\item Download the latest BLCR module v0.6.1 from:\\
	http://ftg.lbl.gov/CheckpointRestart/CheckpointDownloads.shtml
\item Configure and make the module.
\item Install the module
\end{itemize}


\paragraph{Start}
%Describe the actions necessary to begin execution of the procedure.
\begin{itemize}
\item Check that the BLCR modules are inserted properly. {\tt lsmod } will display blcr, blcr\_vmadump and blcr\_imports if they are there.
\item Run {\tt make insmod} command from the installation directory if they are not there already (requires root).
\item Start the Java application using {\tt cr\_run }. The entire command will look something like this:\\
	{\tt cr\_run bin/run.sh}
\end{itemize}

\paragraph{Proceed}
%Describe any actions necessary during execution of the procedure.
\begin{itemize}
\item While the application is running determine its pid using {\tt ps }.
\item Run the BLCR checkpoint command\\
	{\tt cr\_checkpoint [options] PID}\\
	where OPTIONS could be a Linux signal such as the term or kill signal, and the PID used is the one previously determined.
\item Check that the checkpoint file was created in the current directory. It will have be called 'context.PID'.
\item Restart the application from the checkpoint using the command\\
	{\tt cr\_restart context.PID }.
\item Observe that the application finishes correctly.
\end{itemize}

\paragraph{Measure}
%Describe how the test measurements will be made (e.g., describe how remote terminal response time is to be
%measured using a network simulator).

\paragraph{Shut Down}
%Describe the actions necessary to suspend testing, when unscheduled events dictate.
The severity of the error messages will dictate whether the test needs to be suspended.

\paragraph{Restart}
%Identify any procedural restart points and describe the actions necessary to restart the procedure at each of
%these points.
We restart the application from the checkpointed status using the standard process, without a preparation task.

\paragraph{Stop}
%Describe the actions necessary to bring execution to an orderly halt.
The Java application will stop manually by pressing CTRL and C form the shell.

\paragraph{Wrap Up}
%Describe the actions necessary to restore the environment.
We can never reproduce the same environment as when the application was running, because it contains references to kernel identifiers that
have been deleted by the kernel. Those identifiers now means nothing for the kernel and no resource is associated to them. Read and write
operations will fail.

\paragraph{Contingencies}
%Describe the actions necessary to deal with anomalous events that may occur during execution.
Knowing this issue we can write applications that deal with it and allow the Java application to recover from such failures after a
checkpointing. As explained before, one of these actions can be to re\-open and re\-bind server sockets when an IOException is thrown.
Client sockets (threads talking with another server) need not be re-open, because it is probably that the remote server.
